\recette{Pâte à pain}
\ingredient{Pour 2 baguettes}
\ingredient{10g de levure déshydratée}Commencer par placer la levure dans un récipient. Si levure fraîche, doubler la dose et émietter (mais moins recommandé).
\ingredient{400mL d'eau tiède}Ajouter un peu d'eau tiède (prélevée des 400mL). Laisser la levure se dissoudre pendant une dizaine de minutes.
\ingredient{8g de sel}Ajouter le reste de l'eau, puis le sel.
\ingredient{500g de farine (T55 ou T65)}Ajouter la farine en une fois, puis mélanger (si la pâte n'est pas homogène, les grumeaux disparaîtront à la cuisson).
\sidedish{Couvrir le récipient d'un torchon propre et laisser lever 1h30 dans un endroit chaud. Préchauffer le four à 240° avec de l'eau dans le lèche-frite. Verser la pâte dans des moules à baguettes beurrés et farinés, et enfourner pour 25 minutes.}
\astuce{Pour faire lever, un bol d'eau bouillante dans le lèche-frite du four et placer le récipient dans le four fermé. Avant la cuisson, on peut aussi ajouter des graines.}
\end{recipe}
