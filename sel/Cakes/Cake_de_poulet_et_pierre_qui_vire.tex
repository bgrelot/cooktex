\recette{Cake de poulet et pierre qui vire}
\ingredient{1 blanc de poulet}Coupez le blanc de poulet en petits dés.
\ingredient{1 c.s. de curry}Mélangez le poulet au curry.
\ingredient{1 pierre qui vire}Coupez la pierre qui vire en quatre puis détaillez-le en éclats à l'aide de la pointe d'un couteau.
\ingredient{3 \oe ufs}
\ingredient{3 c.s. d'huile d'olive}
\ingredient{10 cL de lait}Dans un saladier, fouettez les \oe ufs avec l'huile et le lait.
\ingredient{180g de farine}
\ingredient{1 sachet de levure}Mélangez la farine et la levure dans un bol puis versez dans le saladier tout en fouettant.
\ingredient{sel, poivre}
\ingredient{50g de raisains blonds}
\ingredient{100g de gruyère râpé}Salez, poivrez, ajoutez les dés de poulet, les raisins, les éclats de pierre qui vire et le gruyère râpé. Mélangez le tout délicatement.
\ingredient{une noix de beurre}Versez la préparation dans un moule à cake préalablement beurré.
\sidedish{Faites cuire au four pendant 45 minutes}.
\hint{La pierre qui vire doit être fraîche. Plus sèche, le goût est plus fort. La recette est pour 4 à 5 personnes.}
\end{recipe}
