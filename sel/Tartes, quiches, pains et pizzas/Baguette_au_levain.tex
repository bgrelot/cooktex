\recette{Baguette au levain}
\ingredient{}\textit{Premier jour : Vérifiez que votre levain est mûr.}
\ingredient{}Le levain est prêt à être utilisé lorsqu'il paraît un peu gonflé, et que des bulles sont apparues à la surface, mais pas sur toute la surface ; voir la photo ci-dessous. Lorsqu'il fait une température normale, mon levain atteint ce stade 6 heures après un repas où je lui ai donné à peu près son poids en farine et son poids en eau.
\ingredient{}
\ingredient{}\textit{Premier jour : Préparez la pâte.}
\ingredient{167 de levain}
\ingredient{500g de farine ou de mélange de farine}
\ingredient{333g d'eau}Dans un grand saladier, ou dans le bol de votre robot-pétrin, mélangez les farines, l'eau, le levain et le gluten, jusqu'à ce que toute la farine soit incorporée. (Je me sers du crochet du KitchenAid pour mélanger à la main d'abord, avant de le fixer au robot que je lance 20 s en vitesse 1, juste pour que tout se mélange ; vous pouvez utiliser un fouet danois ou une simple cuillère en bois.)
\ingredient{}Laissez reposer 20 à 40 minutes. C'est la phase d'autolyse, qui permet à la farine d'absorber un maximum d'eau avant que le grand méchant sel n'arrive.
\ingredient{8.5g de sel gris de Guérande}Ajoutez le sel et pétrissez avec le crochet à basse vitesse pendant 5 minutes. Si vous travaillez à la main et que vous trouvez la pâte trop collante pour la pétrir, vous pouvez vous contenter de la plier sur elle-même avec la corne, comme sur cette vidéo, pendant 7 minutes.
\ingredient{}
\ingredient{}\textit{Premier/deuxième jour : Laissez fermenter la pâte.}
\ingredient{}Couvrez le bol d'un torchon et laissez reposer la pâte à température ambiante pendant 1 heure. Au bout d'une heure, pliez la pâte (comme sur cette vidéo) une douzaine de fois. Ceci permet de développer les arômes et d'obtenir une belle mie. Remettez le torchon.
\ingredient{}Laissez reposer 1 heure de plus et pliez la pâte à nouveau comme indiqué ci-dessus.
\ingredient{}A ce stade, je verse la pâte dans un bol plus petit -- de 2 litres de contenance -- parce que le bol de mon KitchenAid ne tient pas dans mon frigo, mais ce n'est pas obligatoire.
\ingredient{}Déposez un morceau de film plastique sur la pâte (il doit toucher la pâte).
\ingredient{}
\ingredient{}\textit{Deuxième jour : Façonnez les baguettes.}
\ingredient{}Retirez le bol du réfrigérateur. La pâte devrait avoir à peu près doublé de volume.
\ingredient{}Retirez le plastique et remplacez-le par un torchon. Laissez la pâte revenir à température environ 1 heure.
\ingredient{}Mettez une pierre à pizza carrée ou rectangulaire à mi-hauteur dans le four. Préchauffez-le à 300°C (ou, si votre four ne monte pas si haut, à sa température maximale) pendant 30 minutes. Si vous n'avez pas de pierre à pizza, préchauffez le four à 240°C et préparez une plaque à pâtisserie que vous recouvrirez de papier sulfurisé si elle n'est pas anti-adhésive.
\ingredient{}Préparez un torchon de lin que vous réserverez pour la boulange et farinez-le bien. On appelle ce tissu la couche. (Il ne sera pas nécessaire de le laver ensuite ; plus vous l'utiliserez, plus il sera saturé de farine, moins la pâte pourra attacher.)
\ingredient{}Renversez la pâte sur une surface généreusement farinée en vous aidant de la corne pour la faire sortir du bol (j'utilise un vieux tapis de cuisson en silicone bien fariné). Divisez la pâte en 4 pâtons de tailles similaires ; ce n'est pas évident à faire à l'oeil, mais vous pouvez corriger à la balance après si vous voulez.
\ingredient{}Façonnez chaque pâton en un boudin, comme dans la première partie de cette vidéo. Laissez reposer 5 à 10 minutes.
\ingredient{}Roulez chaque boudin sur lui-même pour en allonger la forme, sans dépasser la largeur de votre pierre à pizza (ou plaque de cuisson). Après avoir façonné chaque baguette, déposez-la sur la couche, en formant un pli dans le tissu de part et d'autre pour soutenir la forme (voir la photo ci-dessous). Couvrez d'un torchon et laissez reposer pendant le temps de préchauffage restant.
\ingredient{}
\ingredient{}\textit{Deuxième jour : Donnez "un coup de buée".}
\ingredient{}Pendant les 5 dernières minutes de préchauffage, insérez une lèche-frite (= une plaque de four avec rebords) à l'étage le plus bas du four. Amenez environ 400 ml d'eau à ébullition dans la bouilloire. Juste avant d'enfourner les baguettes, assurez-vous d'avoir un vêtement à manches longues, mettez une manique épaisse, et versez la moitié de l'eau bouillante dans la lèche-frite -- l'eau va bouillir et dégager une grande quantité de vapeur, c'est un peu impressionnant -- avant de refermer rapidement la porte du four.
\ingredient{}Ceci va créer un environnement bien humide qui favorisera la formation d'une belle croûte. Faites vraiment attention de ne pas vous brûler (d'où les manches longues et la manique) et pour cette étape, faites sortir de la cuisine les enfants et les animaux familiers.
\ingredient{}
\ingredient{}\textit{Deuxième jour : Incisez et enfournez les baguettes.}
\ingredient{}Si vous utilisez une pierre à pizza, placez 2 des baguettes sur une pelle à pizza bien farinée ; si les pâtons ne sont pas tout à fait de la même taille, commencez par les deux plus gros. Faites trois incisions sur le dessus d'un geste déterminé, en tenant la lame à un angle de 45°. Déposez-les sur la pierre à pizza, en travaillant rapidement (mais sans précipitation) pour que le four reste ouvert le moins de temps possible.
\ingredient{}Recommencez avec les 2 autres baguettes. Versez le reste de l'eau bouillante dans la lèche-frite et baissez la température à 220°C.
\ingredient{}Si vous n'avez pas de pierre à pizza, mettez les 4 baguettes sur la plaque de cuisson. Incisez-les comme indiqué plus haut et enfournez à mi-hauteur. Versez le reste de l'eau dans la lèche-frite, mais ne baissez pas la température.
\ingredient{}
\sidedish{Laissez cuire 20 à 25 minutes, en permutant les baguettes au bout de 15 minutes de telle sorte que celles qui étaient dans le fond du four soient plus près de la porte et vice versa, jusqu'à ce qu'elles soient brun-doré sur le dessus et le dessous. Lorsqu'on tape au-dessous, ça doit sonner creux. Si la couleur est bonne mais que vous craignez qu'elles ne soient pas tout à fait cuites à coeur, vous pouvez les laisser 5 à 10 minutes de plus dans le four éteint. Laissez refroidir une heure sur grille avant de les goûter.}
\ingredient{}\textit{source : cnz.to}
\end{recipe}
