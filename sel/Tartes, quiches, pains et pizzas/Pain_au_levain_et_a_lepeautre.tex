\recette{Pain au levain et à l'épeautre}
\ingredient{D'après Eric Kayser}
\ingredient{245g de farine de blé T65}
\ingredient{130g de farine d'épeautre T110}
\ingredient{215g d'eau à 20°C}
\ingredient{110g de levain liquide}
\ingredient{8g de sel fin gris de Guérande}Dans le bol du robot pétrisseur, verser les farines, l'eau, le levain, la levure et le sel (le sel ne doit pas toucher le levain ni la levure). Pétrir à vitesse moyenne pendant 4 minutes puis à vitesse rapide pendant 5 minutes.
\ingredient{}Laisser la pâte dans le bol, fomer une boule et recouvrir le bol d'un torchon légèrement humide.
\ingredient{}Laisser reposer 1h30 à une température de 25-30°C. La pâte doit avoir gonflé.
\ingredient{}Verser la pâte sur un plan de travail fariné. Former une boule et la déposer sur un torchon fariné. Recouvrir la boule d'un torchon et laisser reposer 30 minutes.
\ingredient{}Aplatir doucement la boule avec la paume de la main. Ramener les bords vers le centre et reformer une boule. Faire rouler la boule entre les mains pour obtenir une forme régulière.
\ingredient{}Placer la boule (soudure au dessus) à nouveau sur un linge fariné.
\ingredient{}Recouvrir la boule d'un torchon légèrement humide et laisser pousser à nouveau 1h30. La boule va à nouveau devenir plus volumineuse.
\ingredient{}Positionner un lèche-frite dans le bas du four. Préchauffer le four à 230°C en chaleur tournante. Retourner la boule (soudure en dessous) sur une plaque recouverte de papier sulfurisé. Faire 6 entailles sur le dessus du pain (à la lame de rasoir ou aux ciseaux).
\sidedish{Dès que la température du four est atteinte, vers 1 verre d'eau dans le lèche-frite et enfourner aussitôt le pain ; faire cuire 35 minutes. A la sortie du four, déposer la boule de pain sur une grille à pâtisserie de façon à ce que l'humidité puisse s'échapper également par le dessous.}
\hint{Compter environ 370g de farine en tout, avec jusqu'à 2/3 d'épeautre. Plusieurs combinaisons fonctionnent bien : 1/3 T65, 1/3 T80, 1/3 épeautre ; 1/2 épeautre, 1/2 T80 ; 320g de seigle, 50g de sarrasin ; 300g de seigle, 70g d'épeautre.}
\ingredient{}Pour faire de plus jolis pains, on fait gonfler 1h, puis \textit{stretch and fold}. On fait à nouveau gonfler 1h puis \textit{stretch and fold}. On met à reposer au moins 12h au frigo. En le resortant, on le laisse reposer 1h à température ambiance, un dernier \textit{stretch and fold}, puis on préchauffe le four et on cuit.
\ingredient{}Dans le cas du seigle, pas de \textit{stretch and fold} : on fait une boule grossière, on laisse à température ambiante pendant 13 à 15h, on met sous une cloche à pain et au four.
\ingredient{}Pour tous les pains, faire cuire 50 minutes dans la cloche et 5 minutes sans la cloche. Il vaut mieux préchauffer à 250°C et faire cuire à 225°C. S'il n'y pas de cloche, ne pas oublier de mettre de l'eau dans le lèche-frite.
\end{recipe}