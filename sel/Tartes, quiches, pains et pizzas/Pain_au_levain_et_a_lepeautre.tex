\recette{Pain au levait et à l'épeautre}
\ingredient{D'après Eric Kayser}
\ingredient{245g de farine de blé T65}
\ingredient{130g de farine d'épeautre T110}
\ingredient{230g d'eau à 20°C}
\ingredient{110g de levain liquide}
\ingredient{8g de sel fin gris de Guérande}Dans le bol du robot pétrisseur, verser les farines, l'eau, le levain, la levure et le sel (le sel ne doit pas toucher le levain ni la levure). Pétrir à vitesse moyenne pendant 4 minuets puis à vitesse rapide pendant 5 minutes.
\ingredient{}Laisser la pâte dans le bol, fomer une boule et recouvrir le bol d'un torchon légèrement humide.
\ingredient{}Laisser reposer 1h30 à une température de 25-30°C. La pâte doit avoir gonflé.
\ingredient{}Verser la pâte sur un plan de travail fariné. Former une boule et la déposer sur un torchon fariné. Recouvrir la boule d'un torchon et laisser reposer 30 minutes.
\ingredient{}Aplatir doucement la boule avec la paume de la main. Ramener les bords vers le centre et reformer une boule. Faire rouler la boule entre les mains pour obtenir une forme régulière.
\ingredient{}Placer la boule (soudure au dessus) à nouveau sur un linge fariné.
\ingredient{}Recouvrir la boule d'un torchon légèrement humide et laisser pousser à nouveau 1h30. La boule va à nouveau devenir plus volumineuse.
\ingredient{}Positionner un lèche-frite dans le bas du four. Préchauffer le four à 230°C en chaleur tournante. Retourner la boule (soudure en dessous) sur une plaque recouverte de papier sulfurisé. Faire 6 entailles sur le dessus du pain (à la lame de rasoir ou aux ciseaux).
\sidedish{Dès que la température du four est atteinte, vers 1 verre d'eau dans le lèche-frite et enfourner aussitôt le pain ; faire cuire 35 minutes. A la sortie du four, déposer la boule de pain sur une grille à pâtisserie de façon à ce que l'humidité puisse s'échapper également par le dessous.}
\end{recipe}
