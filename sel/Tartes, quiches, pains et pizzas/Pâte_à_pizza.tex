\recette{Pâte à Pizza}
\ingredient{pour 2 pizzas}
\ingredient{500 g. de farine}On commence par tamiser la farine pour avoir une préparation bien fluide.
\ingredient{1 cuillère à café et demie de sel}Creuser un puits au centre de la farine et y mettre le sel.
\ingredient{2 sachets de levure de boulanger}Intégrer la levure ; attention : certaines levures doivent être délayées dans de l'eau ou du lait tiède pendant 15 minutes avant intégration
\ingredient{2 cuillères à soupe d'huile (d'olive)}
\ingredient{1 cuillère à soupe de lait chaud}Ajouter l'huile et le lait (sauf si le lait a servi à délayer la levure).
\ingredient{3 dL d'eau tiède}Mettre par petites quantités l'eau et pétrir. Tant que la pâte est farineuse, on ajoute de l'eau et on continue à pétrir. Quand la pâte forme une jolie boule et ne colle pas au plan de travail, on appuye avec le doigt sur la surface. Si la pâte remonte, c'est gagné, la pâte est prête. On peut ensuite la diviser en 2 pâtons, on repétrit un peu chacun avant de les mettre dans des saladiers.
\ingredient{}Fariner légèrement le dessus des pâtons et laisser monter au chaud avec un torchon humide sur les saladiers (ou ajouter quelques gouttes d'eau toutes les heures pour ne pas que les pâtons forment une croûte).
\ingredient{}Avant de garnir, on récupère chaque pâton, on pétrit en étirant et repliant jusqu'à avoir une bonne consistence (\textit{stretch and fold}).
\sidedish{Cuire au four à 240°C 15 à 20 minutes}
\astuce{Utiliser de la farine fluide, ca va beaucoup plus vite pour mélanger. On peut également remplacer une partie de la levure par du levain (environ 80g par pizza).}
\end{recipe}
