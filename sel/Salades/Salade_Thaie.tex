\recette{Salade Thaïe}
\ingredient{pour 6 pers.}
\ingredient{1 morceau de faux-filet ou de rumsteack paré de 650g}
\ingredient{1 cs d'huile}Badigeonnez la viande avec un peu d'huile et faites-la cuire de tous les côtés sous le gril du four, au barbecue ou à la poële 12 à 15 minutes. Après cuisson, laissez reposer la viande 5 minutes puis coupez-la en fines lamelles.
\ingredient{150g de jeunes pousses de salade}
\ingredient{4 oignons nouveaux}Nettoyez la salade, pelez et émincez les oignons.
\ingredient{1 combava (ou 1 citron vert)}Rincez et essuyez le combava, râpez son zeste.
\ingredient{3 piments rouges longs}Epépinez et émincez les piments.
\ingredient{1 bouquet de coriande}
\ingredient{1 bouquet de menthe}
\ingredient{1 bouquet de basilic}Effeuillez toutes les herbes.
\ingredient{pour la sauce :}
\ingredient{1 cc de sauce soja}
\ingredient{2 cs de sauce nuoc-mâm}
\ingredient{2 cs de jus de citron vert}
\ingredient{2 cs de sucre roux}
\ingredient{1 cs d'huile}Dans un bol, mélangez tous les ingrédients de la sauce.
\ingredient{}Dans un saladier, mélangez la salade, les oignons, le zeste de combava, le piment et les herbes. Ajoutez les lamelles de boeuf et la sauce. Mélangez. Servez frais accompagné d'une sauce soja.
\hint{Accorder avec un bordeaux rosé.}
\hint{François-Régis Gaudry propose une variante où on fait mariner la viande 1 heure à température ambiante avant de la faire saisir 3 minutes dans une sauteuse (avec la sauce). On enlève la viande, on garde la sauce pour la faire refroidir et elle viendra assaisonner la salade. Il ajoute aussi des cacahuètes concassées sur le dessus de la salade.}
\end{recipe}
