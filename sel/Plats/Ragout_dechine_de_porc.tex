\recette{Ragoût d'échine de porc aux pommes de terre}
\ingredient{pour 4 à 6 pers.}
\ingredient{900g d'échine de porc désossée} Détaillez la viande en cubes.
\ingredient{2 carottes}
\ingredient{2 oignons}
\ingredient{4 échalotes}
\ingredient{1 tête d'ail}Epluchez l'ail, les oignons, les échalotes et les carottes. Ciselez les oignons et les échalottes, coupez les carottes en petits tronçons, gardez les gousses d'ail entières.
\ingredient{6 capsules de cardamome}Récupérez les graines contenues dans les capsules de cardamome.
\ingredient{1 cs rase de graines de cumin}Faites chauffer une cocotte allant au four, et faites griller les graines de cumin et cardamome à sec, pendant 1 à 2 minutes, en mélangeant.
\ingredient{6 cs d'huile d'olive}Versez l'huile d'olive, ajoutez les oignons et les échalotes, puis laissez fondre 5 minutes.
Ajoutez la viande, mélangez et laissez dorer 5 minutes.
\ingredient{1 cc rase de cumin en poudre}
\ingredient{1 cc rase de cannelle}
\ingredient{1 cc rase de gingembre en poudre}
\ingredient{2 cs rases de farine}
\ingredient{sel et poivre}Ajoutez ensuite les gousses d'ail et les carottes, saupoudrez de cumin, de cannelle, de gingembre et de farine, salez et poivrez.
\ingredient{50cL de bouillon de volaille}Versez le bouillon tout en mélangeant. Gardez au chaud.
\ingredient{500g de pomme de terre roseval}Préchauffez le four à 170°C. Coupez les pommes de terre en tranches assez fines et disposez-les en rosace assez serrée sur la viande.
\ingredient{5 pincées de thym}
\ingredient{30g de beurre}Salez, poivrez, ajouez le thym et le beurre en parcelles.
\sidedish{Couvrez et enfournez pour 1h30 de cuisson. Augmentez ensuite la température du four à 220°C, retirez le couvercle et poursuivez la cuisson 15 minutes. Servez le ragoût sans attendre.}
\hint{Associer à un Corbières rouge.}
\end{recipe}
