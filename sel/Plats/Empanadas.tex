\recette{Empanadas}
\ingredient{}Pour une douzaine d'empanadas
\ingredient{La pate}
\ingredient{12 cl d'eau}Dans une casserole, 
\ingredient{1/2 verre d'huile d'olive}mettre à chauffer l'eau et l'huile avec le sel.
\ingredient{une bonne pincée de sel} 
\ingredient{300g de farine} Lorsque le mélange bout, le retirer du feu et ajouter la farine en une seule fois
\ingredient{1 œuf} Lorsque la pâte forme un mélange homogène, incorporer l'œuf.
\ingredient{}Pétrir un peu la pate et la mettre de côté.
\ingredient{}Acheter de la pate brisée ça marche aussi, mais c'est moins drôle.
\ingredient{}
\ingredient{La garniture}
\ingredient{}On peut mettre tout ce qui traine : épinards, fromage, viande hachée avec des oignons, maïs, champignons...
\ingredient{}
\ingredient{}Une fois la pate étalée (il faut qu'elle soit assez fine), la découper en ronds 
\ingredient{}Déposer la garniture sur la pâte et refermer.
\ingredient{1 jaune d'œuf}Dorer chaque chausson au jaune d'œuf
\sidedish{Cuire ensuite la fournée d'empanadas au four pendant une vingtaine de minutes, à 180°}
\hint{Si les empanadas ont été faites avec soin, elles sont relativement pratiques à transporter (pour changer un peu des sandwichs...)}
\end{recipe}
