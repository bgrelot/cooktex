\recette{Epaule d'agneau de 7 heures d'Alain Ducasse}
\ingredient{pour 4 à 5 personnes}
\ingredient{1 épaule d'agneau}Nettoyer et ficeler l'épaule.
\ingredient{sel, poivre}Saler, poivrer et réserver à température ambiante.
\ingredient{1 oignon}
\ingredient{1 carotte}Emincer les oignons et couper la carotte en petits cubes.
\ingredient{1 tête d'ail}Eplucher les gousses d’ail.
\ingredient{2 tranches de lard épaisses, ou morceau de lard d'arnad}Détailler le lard en tronçons.
\ingredient{25cL de fond de veau}Préparer votre fond de veau.
\ingredient{2cs d'huile d'olive}Faire chauffer 2 cs d'huile d'olive dans une cocotte en fonte, faire dorer l'épaule d’agneau sur tous les côtés à feu vif pendant 5 min. Retirer l'agneau et réserver.
\ingredient{}Mettre l'ail, l'oignon, la carotte et le lard dans la cocotte et faire revenir 2/3 min à feu moyen en remuant.
\ingredient{10cL de vin blanc sec}Ajouter le vin blanc, mélanger et laisser évaporer presque totalement.
\ingredient{1 bouquet garni}Remettre l'épaule d'agneau sur les légumes, ajouter le fond de veau et le bouquet garni.
\sidedish{Après avoir préchauffé à 120°C, enfourner pour 7 heures.}
\hint{Dans certains cas, peut être intéressant de faire une pâte morte autour de la cocotte.}
