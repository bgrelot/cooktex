\recette{Velouté de chou-fleur au cidre}
\ingredient{Pour 4 personnes : 1/2 chou-fleur}Retirez les feuilles vertes du chou-fleur. Coupez les inflorescences et jetez les trognons. Trempez-les dans de l'eau vinaigrée et égouttez.
\ingredient{1 pomme de terre (150 g.)}Pelez la pomme de terre et coupez-la en cubes égaux.
\ingredient{1 petit oignon}Pelez et émincez l'oignon.
\ingredient{50 g. de beurre demi sel}Dans une grande casserole, faites chauffer la moitié du beurre et faites-y fondre les oignons pendant 5 minutes.
\ingredient{1 tasse de cidre}Versez 70 cL d'eau et 1 tasse de cidre (20 cL). Portez à ébullition, salez et ajoutez les légumes. Laissez cuire 35 minutes.
\ingredient{1/2 pain de seigle}Faites fondre le reste de beurre dans la poêle et faites dorer le pain de seigle coupé en dés.
\ingredient{1 jaune d'\oe uf}
\ingredient{2 cs de crème fraîche liquide}Mixez le velouté, ajoutez le jaune d'\oe uf et la crème. Rectifiez l'assaisonnement et servez avec les croûtons.
\hint{Pour aller plus vite, utilisez du chou-fleur surgelé. Dégustez avec un Cidre d'automne.}
\end{recipe}
