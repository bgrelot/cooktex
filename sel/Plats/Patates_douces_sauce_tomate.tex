\recette{Patates douces à la sauce tomate, citron vert et cardamome}
\ingredient{}Pour 4 personnes, en plat principal
\ingredient{}Ce plat a eu plusieurs vies. Il a commencé comme un kefta de maquereau dans une sauce à base de tomate, de citron vert et de cardamome dont nous sommes tombés amoureux. Il est s'est ensuite transformé en boulettes végétaliennes servies dans la même sauce. Finalement, aprés un nombre honteusement élevé d'essais et après que chaque membre de l'équipe s'en est mêélé, nous avons convenu que la patate douce grillée était la meilleure solution. Le mariage est très réussi, mais la sauce sera tout aussi délicieuse avec des pois chiches, du tofu, du poisson ou du poulet. Servez du riz ou du couscous à côté.
\ingredient{}Préchauffez le four à 240 °C (th. 8) en mode chaleur tournante.
\ingredient{4-5 grosses patates douces, avec la peau, coupées en tranches de 2,5cm d'épaisseur}
\ingredient{2 cs d'huile d'olive}
\ingredient{1,5 cs de sirop d'érable}
\ingredient{1/2 cc de cardamome moulue}
\ingredient{1/2 cc de cumin moulin}
\ingredient{sel et poivre noir}Dans un grand bol, mélangez les tranches de patate douce avec l'huile, le sirop d'érable, la cardamome, le cumin, ½ cuil. à café de sel et un bon tour de moulin à poivre. Étalez-les sur une grande plaque tapissée de papier sulfurisé, couvrez de papier d'aluminium et enfournez pour 25 min. Retirez le papier d'aluminium et poursuivez la cuisson pendant 10-12 min, jusqu'à ce que la patate douce soit cuite à point et que la base soit très brunie (cela peut prendre plus de temps si vos tranches sont épaisses et inversement, donc surveillez la cuisson).
\ingredient{7,5cL d'huile d'olive}
\ingredient{6 gousses d'ail, finement hachées (pas écrasées)}
\ingredient{2 piments verts, finement hachés}
\ingredient{2 petites échalotes banane, finement hachées}
\ingredient{400g de tomates en conserve, mixées en coulis lisse}
\ingredient{1 cs de concentré de tomate}
\ingredient{1,5 cc de sucre semoule}
\ingredient{1,5 cc de cardamome moulue}
\ingredient{1 cc de cumin moulu}
\ingredient{2 citrons verts bio : 1 cc de zeste finement râpé, 1 cs de jus, le reste en quartiers pour servir}Pendant la cuisson de la patate douce, préparez la sauce. Mélangez dans une sauteuse l'huile, l'ail, le piment et une pincée de sel, puis faites revenir à feu moyen ce mélange pendant 8-10 min, en remuant de temps en temps, jusqu'à ce que l'ail soit tendre et parfumé (il ne doit pas brunir ou devenir croustillant, réduisez le feu au besoin). Transférez dans un petit bol la moitié de cette huile aromatique et laissez le reste dans la poêle. Ajoutez les échalotes que vous laissez revenir 5 min à feu moyen, en remuant souvent, jusqu'à ce qu'elles soient tendres et translucides. Incorporez les tomates, le concentré de tomates, le sucre, la cardamome, le cumin, le zeste de citron vert et 1 cuil. à café de sel. Laissez cuire encore 5 min en remuant plusieurs fois. Versez 25 d d'eau, portez à petite ébullition, puis laissez frémir 5 min.
\ingredient{}Déposez les patates douces, face brunie vers le haut, dans la sauteuse (elles ne seront pas toutes immergées dans la sauce, mais ce n'est pas grave). Baissez à feu doux, couvrez et poursuivez la cuisson pendant 10 min.
\ingredient{2 cc d'aneth, finement haché, pour servir}Mélangez l'aneth et le jus de citron vert avec l'huile aromatique réservée, puis arrosez les patates douces de cet assaisonnement. Servez dans la poêle, avec les quartiers de citron vert.
\hint{Pour la cuisson sans papier d'alu, plutôt 30 minutes que 12.}
\end{recipe}