\recette{Mafe sénégalais}
\ingredient{pour 4 pers.}
\ingredient{500g de poulet coupé en petits morceaux}
\ingredient{1 oignon haché}
\ingredient{3 gousses d'ail hachées}Dans un saladier, mélanger ensemble les morceaux de poulet, l'oignon et l'ail. 
\ingredient{sel, poivre}Saler et poivrer.
\ingredient{2-3 cs d'huile d'olive}Chauffer l'huile d'olive dans une sauteuse et ajouter les morceaux de poulet (avec l'assaisonnement). Faire revenir environ 5 minutes.
\ingredient{1 cs de gingembre frais râpé}
\ingredient{1 cc de paprika}
\ingredient{1 piment fort}
\ingredient{1/2 cc de thym}Ajouter les épices et les herbes, saler, poivrer et bien mélanger le tout. Laisser revenir jusqu'à ce que l'oignon soit translucide.
\ingredient{2 tomates (75g) coupées en dés}
\ingredient{1 cs de concentré de tomate}
\ingredient{750mL de bouillon de poulet}Ajouter les tomates réduites en purée, le concentré, le bouillon et mélanger le tout.
\sidedish{Couvrir et laisser mijoter environ 20-30 minutes.}
\hint{On peut ajouter des légumes (patates douces, carottes) et laisser mijoter.}
\ingredient{}Avec ou sans légumes, laisser mijoter à découvert 20-30 minutes de plus jusqu'à ce que la sauce devienne plus épaisse.
\ingredient{2 cs de beurre de cacahuète crémeux}Dans un bol, verser le beurre de cacahuète et ajouter 2 louches de la sauce de cuisson. Fouetter jusqu'à ce que le mélange soit homogène, continuer à incorporer la sauce au beurre jusqu'à liquéfier légèrement le mélange.
\ingredient{}Verser la sauce au beurre dans la casserole et mélanger.
\ingredient{}Couvrir à moitié la casserole et cuire jusqu'à ce que l'huile remonte à la surface.
\ingredient{1/2 botte de coriandre ciselée}Retirer du feu et ajouter la coriandre. Parsemer d'une pincée de paprika si désiré.
\hint{Servir accompagné de riz blanc, riz thaï ou couscous.}
\end{recipe}
