\recette{Limousine d’écrevisses}
\ingredient{pour 4 personnes}
\ingredient{24 écrevisses vivantes}Châtrez les écrevisses : tirez la nageoire centrale de la queue en la tournant afin de retirer le boyau noir. Rincez-les et égouttez-les.
\ingredient{2 échalotes grises}Pelez les échalotes et hachez-les menu.
\ingredient{1 feuille de laurier}
\ingredient{1 brin de thym}
\ingredient{6 tiges de persil}Liez les éléments du bouquet garni. 
\ingredient{25g de beurre}Faites fondre le beurre dans une sauteuse de 28 cm. Ajoutez les écrevisses, couvrez et laissez cuire 5 minutes. Ajoutez les échalotes, salez, poivrez et mélangez 2 minutes, sur feu doux.
\ingredient{2cs de Cognac}Versez le cognac et enflammez-le.
\ingredient{1/2L de vin blanc sec}
\ingredient{3cs de purée de tomate}Lorsque la flamme s’est éteinte, ajoutez le vin, la purée de tomate et le bouquet garni. Mélangez 2 minutes puis retirez les écrevisses avec une écumoire et gardez-les au chaud. 
\ingredient{100g de crème fraîche épaisse}Faites réduire le contenu de la sauteuse sur feu vif, pendant 5 minutes environ, jusqu’à obtention d’un liquide sirupeux. Ajoutez la moitié de la crème, laissez cuire encore 2 minutes puis retirez le bouquet garni. 
\ingredient{2 jaunes d'oeufs}
\ingredient{2 pincées de piment de Cayenne}Fouettez les jaunes d’oeufs à la fourchette en y ajoutant le reste de la crème. Versez ce mélange dans la sauteuse et retirez du feu. Tournez avec une spatule jusqu’à obtention d’une sauce onctueuse. Ajoutez les écrevisses et mélangez 30 secondes. 
\ingredient{}Versez les écrevisses et leur sauce dans un plat creux, parsemez d’estragon ciselé et servez aussitôt. 
\end{recipe}