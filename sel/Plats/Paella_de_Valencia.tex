\recette{Paëlla de Valencia}
\ingredient{pour 6 personnes}
\ingredient{18 moules}Nettoyez les moules.
\ingredient{200g de chorizo}Découpez le chorizo en rondelles.
\ingredient{3 oignons}
\ingredient{3 gousses d'ail}Émincez les oignons et les gousses d’ail.
\ingredient{1 poivron rouge}
\ingredient{1 poivron vert}
\ingredient{3 tomates}Débitez les poivrons en dés et les tomates en quartiers.
\ingredient{1 poulet de 1.5kg découpé}
\ingredient{10cL d'huile d'olive}
\ingredient{sel, poivre}Faites revenir les morceaux de poulet avec 3 cuillerées d’huile d’olive dans une cocotte. Lorsqu’ils commencent à se colorer, ajoutez le chorizo, salez et poivrez légèrement puis couvrez et laissez mijoter 20 minutes.
\ingredient{400g de riz long grain}Pendant ce temps, mesurez le volume de riz et préparez une fois et demie ce volume d’eau chaude. Faites chauffer le reste de l’huile d’olive dans un plat spécial Paella (ou une grande sauteuse).
\ingredient{6 langoustines}
\ingredient{18 crevettes bouquet}Faites-y revenir les oignons et les poivrons pendant dix minutes sur feu moyen ; ajoutez ensuite l’ail, le riz, les crevettes et les langoustines.
\ingredient{500g de petits pois frais}
\ingredient{piment de Cayenne}
\ingredient{safran}Remuez bien jusqu’à ce que le riz soit devenu translucide. Ajoutez alors les tomates, le poulet et le chorizo avec leur jus de cuisson ainsi que les petits pois. Assaisonnez de sel, de Cayenne et de safran. 
\ingredient{}Mélangez bien puis mouillez avec l’eau chaude préparée.
\ingredient{}Portez à ébullition et laissez mijoter jusqu’à ce que le riz ait absorbé tout le liquide, pendant vingt minutes environ.
\ingredient{}Enfoncez alors les moules dans le riz. Dès qu’elles sont ouvertes, la paëlla est prête. Servez là aussitôt, dans son plat de cuisson.
\end{recipe}