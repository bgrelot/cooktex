\recette{Epaule d'agneau confite façon tajine de Christian Le Squer}
\ingredient{pour 4 pers.}Préchauffez le four à 130 °C (th. 4-5).
\ingredient{1 épaule d'agneau}
\ingredient{sel, poivre}Salez et poivrez une épaule d’agneau, faites-la colorer dans la cocotte, réservez.
\ingredient{4 oignons}
\ingredient{huile d'olive}Émincez et faites suer 4 oignons dans un filet d’huile d’olive. 
\ingredient{4 gousses d'ail pelées}
\ingredient{1 bouquet garni}
\ingredient{50 g de raisins de Corinthe}
\ingredient{50 g de raisins de Smyrne}
\ingredient{1 cc de cumin}
\ingredient{1 petit bâton de cannelle}
\ingredient{1 citron confit au sel (dans les épiceries libanaises)}
\ingredient{1 botte de coriandre ciselée}Quand ils sont bien colorés, remettez l’épaule d’agneau dans la cocotte, ajoutez les assaisonnements.
\ingredient{1 cs de farine}Ajoutez la farine, faites dorer la viande.
\ingredient{2L d'eau}
\ingredient{4 bouillons-cube}Couvrez de 2L d'eau, ajoutez les bouillons-cube.
\sidedish{Enfournez pour 3 h environ. Au moment de servir, parsemez d’une poignée d’amandes effilées et torréfiées.}
\hint{A servir avec de la semoule de couscous. On peut aussi mettre des pruneaux dans la sauce.}
\end{recipe}
