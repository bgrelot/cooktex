\recette{Tourte berrichonne}
\ingredient{pour 6 pers.}
\ingredient{500g de farine}
\ingredient{10cL d'eau}
\ingredient{250g de beurre}
\ingredient{2 oeufs}Constituez la pâte brisée au préalable. Sur une planche, déposez la farine, salez-la, pétrissez-la rapidement avec le beurre coupé en morceaux et les oeufs incorporés, ajoutez l'eau, laissez reposer 2h.
Partagez la pâte, abaissez-la en deux morceaux, tapissez la tourtière d'une partie.
\ingredient{500g de pommes de terre fermes}
\ingredient{250g d'oignons}Déposez simultanément les pommes de terre et les oignons qui seront au préalable salés et poivrés.
\ingredient{6 branches de cerfeuil hachées}
\ingredient{1 jaune d'oeuf battu}Hacher le cerfeuil sur cet ensemble, recouvrez la tourtière de la deuxième partie, soudez les bords, badigeonnez la surface avec le jaune d'oeuf battu.
\sidedish{Mettez à four moyen pendant 1 heure et quart.}
\astuce{L'auteur recommande en servant, d'agrémenter ce plat de crème fraîche.}
\end{recipe}
