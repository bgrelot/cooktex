\recette{Gratin de potiron}
\ingredient{750 à 800 g. de potiron épluché}Eplucher le potiron, enlever les graines et le couper en dés moyens.
\ingredient{300 g. de pommes de terre}Eplucher les pommes de terre et les couper aussi en dés moyens.
\ingredient{}Faire cuire le tout dans de l'eau bouillante salée pendant 20 minutes, puis laisser bien égoutter. On utilise l'écumoire pour mettre les dés dans la passoire, sinon tout s'écrase...
\ingredient{50 g. de beurre}
\ingredient{Noix de muscade, sel, poivre}Quand c'est encore tiède, on écrase le tout, on incorpore le beurre, le poivre, un peu de sel encore et de la noix de muscade râtée et un peu de gruyère.
\ingredient{2 \oe ufs}Battre les deux \oe ufs en omelette et bien les incorporer à la purée obtenue.
\ingredient{100 g. de gruyère râpé}
\sidedish{On met le tout dans un plat allant au four (type Pyrex), pas besoin de le beurrer. Saupoudrer de gruyère râpé et faire gratiner une vingtaine de minutes à four chaud Th. 7.}
\hint{En accompagnement : jambon ou saucisses.}
\end{recipe}
