\recette{Omelette berrichonne}
\ingredient{pour 4 pers.}
\ingredient{2 poignées d'oseille fraîche}Lavez et équeutez l'oseille, coupez-la en lanières et faites-la étuver à feu modéré.
\ingredient{100g de jambon de pays}
\ingredient{100g de beurre}Coupez les tranches de jambon en dés, faites-les sauter au beurre sans coloration.
\ingredient{8 oeufs}Réunissez le jambon et l'oseille, battez les oeufs en omelette et procédez à sa cuisson.
\ingredient{25g de gruyère râpé}A mi-cuisson, déposez au centre l'oseille et le jambon, parsemez de gruyère râpé. Pliez l'omelette et renversez-la sur un plat ovale allant au four.
\ingredient{3cs de crème fraîche}
\ingredient{noix de muscade}Faites chauffer 2 cuillerées seulement de crème fraîche dans une petite casserole et assaisonnez de quelques coups de râpe de noix de muscade.
\ingredient{}Incisez le centre de l'omelette pour y introduire la crème.
\sidedish{Avec la 3ème cuillerée de crème, nappez la surface de l'omelette. Passez au four très chaud pendant quelques secondes afin de la glacer et servez.}
\end{recipe}
