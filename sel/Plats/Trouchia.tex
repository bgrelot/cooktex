\recette{Trouchia}
\ingredient{8 oeufs}
\ingredient{20g de parmesan}
\ingredient{1/2 botte de basilic}
\ingredient{sel, poivre} Dans un saladier, casser les oeufs, saler, poivrer, y râper le parmesan, effeuiller et ciseler le basilic, puis battre le tout en omelette à la fourchette. Réserver.
\ingredient{1 botte de blettes} Préparer les blettes en ôtant les côtes et en coupant les verts finement.
\ingredient{2cL d’huile d’olive}
\ingredient{1 oignon doux émincé}
\ingredient{1 gousse d’ail dégermée} Dans une poêle, faire revenir dans l’huile d’olive l’oignon ciselé et l’ail émincé.
\ingredient{} Faire tomber ensuite les verts coupés finement.
\ingredient{} Saler, poivrer.
\ingredient{} Quand les blettes sont bien assouplies, ajouter dans la poêle le mélange d’oeufs battus. Egliser l’ensemble et laisser prendre l’omelette en la faisant cuire à feu doux à l’étouffée pendant 10-15 minutes.
\ingredient{} Retourner la trouchia avec une assiette puis la reverser dans la poêle avec un peu d’huile d’olive. Faire cuire de nouveau à l’étouffée pendant quelques minutes. Pour vérifier la cuisson, planter la pointe d’un couteau, elle doit ressortir sèche.
\astuce{La trouchia peut se manger chaude ou froide}
\end{recipe}
