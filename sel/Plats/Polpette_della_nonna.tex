\recette{Polpette della nonna}
\ingredient{400g. de viande hachée}
\ingredient{15 cm de baguette de pain rassi}Enlever la croute (c'est plus facile quand c'est rassi !).
\ingredient{Eau ou lait chauds}Dans un bol, mettre l'eau ou le lait et passer au micro-ondes, puis faire tremper le pain dedans. Quand le pain est bien rempli, presser pour essorer, puis joindre à la viande hachée.
\ingredient{Sel, poivre, noix de muscade}Ajouter tout ça, puis pétrir.
\ingredient{}Laisser la préparation au frigo pendant à peu près une demie heure pour que la viande se prenne bien en masse.
\ingredient{}Préparer les boulettes de viande.
\ingredient{Farine}Déposer de la farine sur le papier alu, puis rouler les boulettes dans la farine. Les stocker quelque part :-)
\sidedish{Dans une poêle avec un peu de matière grasse chauffée, déposer les boulettes délicatement. Quand elles sont dorées, les retirer et les stocker au même endroit que tout à l'heure. Dans la même matière grasse qui est restée dans la poêle, poser un oignon (coupé en tranches fines), puis faire revenir délicatement.}
\ingredient{3 cuillères à café pleines de farine}Verser la farine dans un verre puis compléter avec de l'eau. Quand les oignons sont prêts, y verser ce verre puis vite remuer pour éviter les grumeaux. Saler un peu.
\ingredient{}Laisser mijoter un peu puis remettre les polpette, puis faire cuire à petit feu une vingtaine de minutes.
\end{recipe}
