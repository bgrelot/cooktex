\recette{Terrine de foie gras}
\ingredient{Pour 500g de foie cru}
\ingredient{1 cs de porto}Porto blanc de préférence, pour ne pas colorer le foie.
\ingredient{1 cs d'armagnac ou cognac}
\ingredient{1 cc de vinaigre de Xérès}
\ingredient{1 cc de sucre en poudre}
\ingredient{1 cc de sel fin}
\ingredient{1/2 cc de poivre blanc}
\ingredient{1 pincée de quatre épices}Préparer la marinade, et faire mariner le foie cru pendant 24h, en remuant à intervalle régulier.
\sidedish{Déposer les morceaux de foie à plat directement dans un plat, à 95°C pendant 23 minutes, avant de mettre en terrine. Récupérer l'excédent de gras et le réserver.}
\ingredient{}Laisser quelques jours reposer avec un poids sur la terrine, puis ôter le poids.
\ingredient{}Faire chauffer le gras réservé pour le liquéfier, puis le verser sur le dessus de la terrine pour avoir une apparence bien nette.
\hint{Evidemment, on peut modifier la marinade avec ce qu'on a sous la main, et on peut aussi farcir le foie gras : chorizo, pruneaux... Un bon foie gras se fait à l'imagination !}
\hint{Rhum et vanille, c'est un mélange assez original qui fonctionne parfaitement.}
\hint{Toujours plus simple de travailler avec un foie dénervé. Mais si on n'a pas le choix, couper le foie en deux dans l'épaisseur, à mi-hauteur et tout du long. Ouvrir comme un livre et séparer les deux faces l'une de l'autre. Attention à la température du foie pour dénerver, il faut qu'il soit à température ambiante sinon les veines cassent au lieu de s'extraire. Plonger les foies nettoyés dans une eau additionnée de galçons et gros sel pour enlever les traces de sang, une heure ou deux. Eponger sur du papier absorbant avant de faire mariner.}
\end{recipe}
