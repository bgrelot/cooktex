\recette{Tortilla de patata}
\ingredient{1kg de pommes de terre} Éplucher et couper les patates en morceaux (pas rondelles), et les mettre à cuire dans une poêle
\ingredient{de l'huile, beaucoup d'huile} en les couvrant généreusement d'huile. Si ça ne ressemble pas à une piscine d'huile ça n'ira pas.
\ingredient{1 oignon} Rajouter l'oignon émincé, et couvrez.
\ingredient{} Une fois que les pommes de terre sont cuites, vider l'huile dans un bol et réserver le contenu de la poële dans un saladier.
\ingredient{8 œufs} Battre les œufs comme pour une omelette, saler et poivrer.
\ingredient{} Rajouter les pommes de terres/oignons aux œufs, remuer (doucement sinon ça va faire de la purée)
\ingredient{} Remettre un peu d'huile dans la poële, et faire cuire la tortilla à feu doux
\ingredient{} Une fois le premier côté doré, retourner la tortilla à l'aide d'une assiette et faire cuire l'autre côté.
\ingredient{} Et une fois que c'est cuit, manger :)
\hint{La tortilla se mange sur une tranche de pain, avec de la tomate éventuellement. Elle se mange chaude, tiède ou froide.\\ On peut également y ajouter des légumes (tomates, courgettes...) mais tout de suite ça devient un peu trop sain et équilibré.}
\end{recipe}

