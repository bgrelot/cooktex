\recette{Verrine poireaux et St Jacques}
\ingredient{Pour 4 verrines}
\ingredient{2 poireaux}Nettoyer les poireaux, les couper en brunoise, faire suer dans l'huile 7 minutes à feu doux.
\ingredient{sel, poivre}
\ingredient{20cL de crème fraîche liquide}
\ingredient{4 g. de gélatine}Saler, poivrer, ajouter la crème, mixer au mixer plongeant, ajouter la gélatine qui a trempée dans une grande quantité d'eau froide.
\ingredient{}Passer au chinois
\ingredient{}Laisser refroidir, puis à l'aide d'une poche à douille garnir les verrines avec cette crème.
\ingredient{4 noix de St Jacques}Couper les noix en petits cubes.
\ingredient{1 citron vert}Prélever le zeste du citron à l'aide d'un zesteur, presser le jus.
\ingredient{1 cs d'huile d'olive}Arroser les saint Jacques avec le jus, saler et poivrer, ajouter l'huile et les zestes, bien mélanger.
\ingredient{}Réserver au frais 1 heure.
\ingredient{}Au moment de servir poser le tartare sur la crème de poireau, écraser quelques baies roses, les parsemer sur le tartare. Servir aussitôt.
\ingredient{}Source : deliceyes.
\end{recipe}

