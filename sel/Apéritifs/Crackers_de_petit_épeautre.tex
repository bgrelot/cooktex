\recette{Crackers de petit épeautre}
\ingredient{Pour un gros bol de crackers}
\ingredient{250g de farine de petit épeautre}
\ingredient{50g de quinoa rouge non cuit}
\ingredient{1/2cc de levure}
\ingredient{1cc de sel fin} Verser tous les ingrédients secs dans la cuve d'un robot.
\ingredient{8cL d'huile d'olive bio}
\ingredient{16cL d'eau chaude} Ajouter l'huile, puis l'eau et mélanger.
\ingredient{} Une fois le mélange homogène, le déposer dans un récipient, couvrir d'un linge humide et réserver pendant une heure au frais.
\ingredient{} Fleurer le plan de travail avec la farine et étaler la pâte très finement (1mm) au rouleau. Si elle est collante, c'est normal !
\ingredient{} Préchauffer le four à 160°, chaleur tournante.
\ingredient{} Détailler la pâte avec un couteau en forme de carrés. Disposer ces derniers sur une plaque recouverte d'un tapis de cuisson ou de papier sulfu, et cuire une dizaine de minutes au four. Attention à la coloration !
\ingredient{} Laisser refroidir et déguster seul ou avec une tartinade (houmous, labné, tapenade...).
\end{recipe}

