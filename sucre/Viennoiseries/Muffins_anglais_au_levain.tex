\recette{Muffins anglais au levain}
\ingredient{pour 6 muffins}
\ingredient{150mL de lait}
\ingredient{50g de beurre}Faites tiédir le lait avec le beurre.
\ingredient{375g de farine T65}
\ingredient{100g de levain liquide}
\ingredient{1cc de sel}
\ingredient{1 oeuf}Mettez dans la cuve de votre robot la farine, le levain, le sel et l’œuf. Versez le lait et le beurre fondu et pétrissez vitesse 1 pendant 1 minute puis vitesse 2 pendant 9 minutes. Vous obtenez une boule de pâte assez ferme. Couvrez et laissez lever pendant 1 heure à température ambiante puis 12 h au frais (idéalement la nuit).
\ingredient{}Le lendemain sortez le bol du frigo et laissez remonter en température pendant 1 h (la pâte ne gonfle pas énormément). Transvasez sur un plan de travail légèrement fariné et étalez délicatement la pâte sur 1 à 2 cm d’épaisseur.
\ingredient{Semoule ou farine de maïs (ou épeautre)}Poudrez de farine de maïs et découpez des disques de pâte à l’aide d’un emporte pièce de 10 cm de diamètre. Placez-les sur une plaque et poudrez de farine de maïs l’autre côté. Couvrez d’un torchon et laissez lever à température ambiante pendant 3 heures.
\sidedish{Préchauffez le four à 180°c. Dans une poêle chaude faites cuire les muffins deux par deux pendant 2 à 3 minutes de chaque côté. Cela va gonfler. Disposez-les sur une plaque au fur et à mesure et faites cuire au four pendant 8 à 10 minutes. Laissez tiédir sur une grille.}
\end{recipe}
