\recette{Gâteaux au potiron (journal des femmes)}
\ingredient{600 g. de potiron ou citrouille débarrassée de sa peau et ses graines}Couper les 600 g. de potiron en petits morceaux, les mettre dans un plat creux allant au micro-ondes. Faire cuire 15 minutes, remuer à la cuillère et remettre 15 minutes au micro-ondes.
\ingredient{300 g. de sucre en poudre}Passer le potiron au mixeur, ajouter le sucre. Mélanger bien l'ensemble.
\ingredient{200 g. de farine}
\ingredient{1 sachet de levure}
\ingredient{1 pincée de sel}Ajouter la farine, la levure et le sel.
\ingredient{75 g. de beurre}Ajouter le beurre fondu.
\ingredient{4 \oe ufs}Ajouter les 4 \oe ufs.
\ingredient{2 cs de fleur d'oranger}
\ingredient{1 cs de citron}Pour terminer, ajouter la fleur d'oranger et le citron.
\sidedish{Si vous avez un plat qui va au four, vous pouvez mettre au fond une feuille de papier sulfurisé. Mettre de la pâte aux 2/3 de la hauteur et placer au four chaud à 180°C, pendant 20 à 25 minutes. Quand les gâteaux sont bien dorés, les sortir du four et laisser refroidir. Décorer le gâteau ou les petits gâteaux avec un peu de sucre glace.}
\hint{La fleur d'oranger et le citron sont facultatifs.\\
Si vous avez un peu plus de potiron, ajouter un peu de farine.\\
Dans un moule unique, la cuisson est plutôt de 35 voire 40 minutes, piquer le gâteau pour tester.\\
Ne remplissez pas plus que les 2/3 car le gâteau gonfle pas mal et il paraît qu'un débordement du gâteau dans le four n'est pas très agréable.}
\end{recipe}
