\recette{Muffins myrtille façon Starbucks}
\ingredient{pour 15 muffins}
\ingredient{45g de farine}
\ingredient{2 cs de sucre en poudre}
\ingredient{2 cs de cassonade}
\ingredient{1/4 cc de cannelle}Mélangez tous les ingrédients secs dans un saladier.
\ingredient{30g de beurre froid coupé en morceaux}Ajoutez le beurre coupé en morceaux et sablez la préparation entre vos doigts jusqu'à ce que la pâte s'amalgame et forme de gros morceaux. Réservez.
\ingredient{}Préchauffez votre four à 180°C et préparez votre moule à muffins avec les caissettes en papier.
\ingredient{105g de beurre à température ambiante}
\ingredient{185g de sucre en poudre}Fouettez le beurre et le sucre dans le bol de votre robot.
\ingredient{1 gousse de vanille}
\ingredient{2 oeufs moyen}Ajoutez la vanille puis les oeufs un à un en raclant les bords du bol entre les deux.
\ingredient{360g de farine}
\ingredient{4 cc de levure chimique}
\ingredient{1/2 cc de sel}
\ingredient{250mL de lait}Ajoutez les ingrédients secs (farine, levure, sel) en alternance avec le lait, en commençant et en terminant par les ingrédients secs. Pensez à racler les bords du bol mais mélangez juste ce qu'il faut.
\ingredient{200g de myrtilles fraîches}Ajoutez les myrtilles et remuez délicatement à l'aide d'une maryse pour ne pas les écraser.
\ingredient{}Remplissez complètement vos caissettes et garnissez de crumble de manière égale.
\sidedish{Enfournez pour 20-25 minutes à 180°C, les muffins doivent être dorés et revenir quand on appuie dessus avec le doigt. Laissez-les refroidir sur une grille et servez à température ambiante (jamais de frigo pour les muffins.}
\textit{Auteur : Valérie Décort}
\hint{Si ce sont des petits muffins, doubler le dosage de crumble.}
\end{recipe}
