\recette{Macarons (environ $50$ macarons de $3cm$ de diamètre)}
\ingredient{200 g. de sucre glace}
\ingredient{200 g. de poudre d'amandes} Mixez longuement afin d'obtenir une poudre fine et homogène. Passer au tamis et mettre de côté. Cette poudre s'appelle le "tant pour tant"
\ingredient{200 g. de sucre en poudre}
\ingredient{8 cL d'eau} Dans une casserole, portez l'eau et le sucre en poudre à ébullition. La température ne doit pas dépasser 110°C.
\ingredient{80 g. de blanc d'\oe uf (environ 3)} Montez les blancs en neige pas trop ferme.
\ingredient{} Versez le sucre cuit sur les blancs en un fin filet en laissant tourner le batteur (à faible vitesse) jusqu'à ce que la meringue ait quasiment refroidi. C'est long.
\ingredient{80 g. de blanc d'\oe uf} Mélangez au "tant pour tant" afin d'obtenir une épaisse pâte d'amande.
\ingredient{} Ajoutez le colorant à la pâte d'amande.
\ingredient{} Le macaronage: incorporer une petite quantité de meringue à la pâte d'amande à l'aide d'une spatule souple en un geste régulier du fond vers le haut et des bords vers le centre. Incorporez en une fois le reste de la meringue aussi délicatement.
\ingredient{} Réalisez des petites boules de pâte bien espacées et bien calibrées sur du papier sulfurisé.
\ingredient{} Tapotez doucement la plaque par le dessous afin de faire éventuellement ressortir des petites bulles d'air et d'uniformiser les coques.
\ingredient{} Mettre les plaques de côté pendant une heure environ afin de permettre aux coques de sécher légèrement.
\sidedish{Enfournez 13min à 145°C (15min pour des macarons de 8cm de diamètre)}
\sidedish{} À la sortie du four glisser la feuille de papier sulfurisé sur un plan de travail légèrelent humidifié pour arrêter la cuisson plus rapidement.
\hint{Vous pouvez également utiliser des moules souples pour faire cuire les macarons et ne pas les faire reposer une heure avant de les enfourner.}
\hint{Pour des colorants variés : www.argato.com}
\end{recipe}
