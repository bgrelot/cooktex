\recette{Canelés}
\ingredient{version préparation Albarock, avec cuisson Baillardran (source : Benoît G.)}
\ingredient{pour 22 canelés}
\ingredient{Préparation, 36h à 48h avant (pas 24h, sinon ça gonfle à la cuisson !)}
\ingredient{1L de lait}
\ingredient{1 gousse de vanille}Emincer finement pour récupérer 4 fois plus d'arôme que entière
\ingredient{grosse pincée de sel}
\ingredient{50g de beurre}Vanille, beurre et sel dans le lait, faire chauffer doucement (la vanille doit s'exprimer !) jusqu'à frémissement en laissant infuser.
\ingredient{500g de sucre semoule}
\ingredient{250g de farine}Mélanger farine et sucre parfaitement au fouet (comme ça, pas de grumeau quand on mélange).
\ingredient{}Quand le lait est à 80°C (lait fumant et pas bouillant), on verse le lait sur le mélange farine-sucre en une fois, au travers d'une passoire fine pour récupérer les morceaux de gousse de vanille.
\ingredient{}On écrase les morceaux de vanille pour faire tomber le jus de vanille, le plus fort.
\ingredient{}On mélange bien, toujours au fouet.
\ingredient{6 jaunes d'\oe uf}
\ingredient{2 \oe ufs entiers}
\ingredient{40mL de rhum (Negrita)}On ajoute tous les \oe ufs (entiers + jaunes), et on mélange encore, puis on ajoute le rhum quand la pâte a déjà bien refroidi.
\ingredient{}On la laisse 36h au frigo pour que la pâte rentre en maturation (pour qu'ils ne gonflent pas). Elle sera alors plus liquide. On peut congeler la pâte crue.
\ingredient{}
\ingredient{Cuisson, le jour-même}
\ingredient{}Sortir la pâte, bien la mélanger et mettre dans les moules 30 minutes avant d'enfourner. Remplir les moules à 1cm du bord (important de laisser, car cela monte beaucoup !) : <0,6L pour remplir 9 moules.
\ingredient{}Four préchauffé 220°C, chaleur tournante. Moules à canelés bien graissés à la bombe depuis 4 côtés (pas trop non plus pour éviter que cela déborde...).
\ingredient{}Bien séparer les moules sur la grille (pas une plaque, pour la répartition de la chaleur) au deuxième cran en partant du bas.
\sidedish{8 minutes à 220°C, puis 50 minutes à 160°C}(15 minutes à 220°C puis 45 minutes à 160°C semble trop cramé).
\ingredient{}Démouler quand bien chaud, dès la sortie du four, en tapant à l'envers sur la grille.
\hint{Enlever le surplus de graisse dedans et au-dehors des moules au Sopalin, pour la prochaine fois. Si la fois précédente a mal fonctionné, brûler les moules avec de la graisse pendant 1h avant, puis re-graisser et essuyer (si moules en cuivre).}
\hint{Pour des mini canelés, 50 minutes à 200°C. Pour les gros (possiblement aussi pour les petits), on peut faire 10 minutes à 230° et 40 à 170°.}
\end{recipe}
