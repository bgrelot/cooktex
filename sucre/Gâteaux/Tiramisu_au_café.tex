\recette{Il Tiramisu (recette de Nonni)}
\ingredient{3 oeufs} Séparer jaunes et blancs des oeufs.
\ingredient{2cs de sucre}
\ingredient{1 sachet de sucre vanillé} Faire blanchir les jaunes avec le sucre.
\ingredient{250g de mascarpone} Ajouter le mascarpone, déjà travaillé avec une cuillère en bois.
\ingredient{10mL de Marsala}
\ingredient{8mL de Cognac} Ajouter le Marsala et le Cognac.
\ingredient{} Monter les blancs en neige bien fermes et les incorporer délicatement à la crème.
\ingredient{1 grande tasse de café fort}
\ingredient{200g de biscuits à la cuillère} Verser le café dans un bol et y tremper les biscuits un par un.
\ingredient{} Disposer progressivement les biscuits dans un plat allant au four à bords hauts.
\ingredient{} Former une première couche, étaler la moitié de la crème au-dessus et recouvrir d'une autre couche de biscuits puis de crème.
\ingredient{} Laisser reposer au réfrigérateur pendant 6 à 8 heures.
\ingredient{Cacao amer} Saupoudrer au moment de servir.
\hint{Pour une variante à la fraise, remplacer le Cognac et le Marsala par du rhum et de la liqueur de fraise. Ajouter 500g de fraises par dessus les deux couches de biscuits pour lesquels on remplace la café par un sirop avec 130g d'eau, 60g de sucre, 3cL de rhum et du sirop de vanille : on porte à ébullition l'eau et le sucre, puis on ajoute le reste.}
\end{recipe}
