\recette{Brookies}
\ingredient{}Préparation de la partie brownie : préchauffez le four à 180°C.
\ingredient{125g de chocolat noir}
\ingredient{75g de beurre}Faites fondre le chocolat et le beurre au bain-marie ou au micro-onde et mélangez bien.
\ingredient{125g de sucer}
\ingredient{2 oeufs}
\ingredient{75g de farine}
\ingredient{50g de noix de pécan}Hors du feu ajoutez le sucre, les œufs, la farine et enfin les noix de pécan en mélangeant bien entre chaque ajout. Versez dans un moule carré ou un cadre d’environ 20 cm de côté
\ingredient{120g de beurre mou}
\ingredient{135g de sucre roux}Préparez ensuite la pâte à cookie : fouettez ensemble le beurre et le sucre roux jusqu’à ce que le mélange devienne crémeux.
\ingredient{1 oeuf}Ajoutez l’œuf et mélangez bien.
\ingredient{165g de farine}
\ingredient{1cc rase de levure chimique}
\ingredient{110g de grosses pépites de chocolat}Incorporez enfin la farine, la levure ainsi que les grosses pépites de chocolats et mélangez rapidement sans trop travailler la pâte.
\ingredient{petites pépites de chocolat}Disposez grossièrement sur l’appareil à brownie. Parsemez le tout de petites pépites de chocolat.
\sidedish{Enfournez pour 20 à 40 mn (cela dépend des fours, mais pas plus surtout ! Sinon l’ensemble sera trop cuit/sec et vous n’aurez pas les textures désirées) jusqu’à ce que la surface soit très légèrement doré. Laissez refroidir complètement avant de détailler des petits carrés.}
\hint{Pour un maximum de gourmandise, dégustez le brookie encore tiède accompagné de glace à la vanille, extase gustative assurée ! La bonne durée est probablement autour de 30-35 minutes.}
\ingredient{}\textit{via chocolatetcaetera.fr}
\end{recipe}
