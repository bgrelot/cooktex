\part*{Recettes salées}
\addcontentsline{toc}{part}{Recettes salées}
\chapitre{Apéritifs}
\recette{Gougères}
\ingredient{325 g. d'eau}
\ingredient{125 g. de sel} Faire bouillir l'eau et le beurre (et saler).
\ingredient{250 g. de farine} Quand le tout bout, retirer du feu et mettre la farine d'un seul coup.
\ingredient{} Bien mélanger et mettre sur le feu jusqu'à ce que la boule se décolle de la casserolle.
\ingredient{5 \oe ufs} Hors du feu, mettre les \oe ufs un par un.
\ingredient{250 g. de gruyère râpé} Ajouter le gruyère et bien mélanger.
\ingredient{} Beurrer une plaque et déposer les choux.
\ingredient{Un jaune d'œuf} Passer le jaune d'œuf pour faire dorer (facultatif).
\sidedish{} Thermostat 6 pendant 30 à 45 minutes.
\end{recipe}

\barre
\recette{Toasts tartiflette}
\ingredient{Pain baguette}Couper en tranches fines.
\ingredient{Reblochon}Déposer un carré de reblochon sur chaque tranche.
\ingredient{Lardons}Déposer un ou deux lardons sur chaque tranche.
\ingredient{Oignon}Déposer une rondelle d'oignon sur chaque tranche.
\ingredient{Origan}Parsemer d'origan.
\sidedish{Passer au four jusqu'à faire fondre le reblochon.}
\end{recipe}

\chapitre{Salades}
\recette{Taboulé}
\ingredient{500 g. de semoule} Verser la semoule dans un grand saladier
\ingredient{1 dL d'huile d'olive} Ajouter l'huile
\ingredient{4 gros citrons} Ajouter le jus de citron puis remuer
\ingredient{2 oignons}
\ingredient{Menthe, persil, ciboulette, estragon, sel et poivre} Hacher les oignons et les herbes sauf la menthe.
\ingredient{1 kg de tomates}
\ingredient{1 kg de concombres} Assaisonner et mélanger en égrainant à la fourchette ; ajouter les tomates et les concombres.
\ingredient{} Laisser au frigo 12 heures (en remuant 2 ou 3 fois pendant)
\ingredient{} Une heure avant de servir, ajouter la menthe et décorer.
\end{recipe}
\chapitre{Sauces}
\chapitre{Cakes}
\recette{Cake au persil et à la moutarde}
\ingredient{6 œufs}
\ingredient{3 cuillières à soupe de moutarde}
\ingredient{Persil selon les goûts}
\ingredient{200 g de jambon blanc} Mélanger les œufs, le persil, la moutarde et le jambon
\ingredient{150 g de gruyère râpé} Ajouter le gruyère
\ingredient{200 g de farine}
\ingredient{1 sachet de levure} puis la farine et la levure. Il est impératif de bien tout mélanger. Saler et poivrer à votre convenance. 
\sidedish{Beurrer un moule à cake, verser la préparation, cuire 40 minutes à 170 degrés (thermostat 5-6)}
\end{recipe}
\barre
\recette{Cake aux lardons}
\ingredient{200 g. de farine}
\ingredient{4 œufs} Ajouter un à un les œufs à la farine
\ingredient{8 cuillières à soupe de lait}
\ingredient{1/2 sachet de levure}
\ingredient{75 g. de beurre fondu} Y ajouter le lait, la levure et le beurre
\ingredient{150 g. de gruyère rapé} 
\ingredient{200 g. de lardons fumés} Terminer par ce qui donne un peu de goût :)
\sidedish{Beurrer un moule à cake et laisser cuire 45 minutes à 180° (Th. 6)}
Les qualités des lardons est relativement importantes, des mauvais lardons très salés se ressentiront. 
\end{recipe}
\barre
\recette{Cake aux poivrons grillés}
\ingredient{250 g farine} Mélanger l'huile, les œufs, le lait, le sel, le poivre et les herbes de Provence.
\ingredient{1 sachet levure} Ajouter la farine et la levure.
\ingredient{4 œufs}
\ingredient{100 g gruyère râpé}
\ingredient{un bocal de 290 g de poivrons grillés à l'huile} Ajouter le gruyère, les poivrons coupés en petites lanières.
\ingredient{10 cl huile (l'huile des poivrons)} 
\ingredient{15 cl lait} 
\ingredient{herbes de Provence}
\ingredient{sel}
\ingredient{poivre} Bien mélanger, versez la préparation dans un moule à cake.
\sidedish{Cuire 1 heure à 180°C.} 
\hint{Vous pouvez aussi faire griller les poivrons vous mêmes, mais c'est tellement bien fait en boîte} 
\end{recipe}
\barre
\recette{Cake de poulet et pierre qui vire}
\ingredient{1 blanc de poulet}Coupez le blanc de poulet en petits dés.
\ingredient{1 c.s. de curry}Mélangez le poulet au curry.
\ingredient{1 pierre qui vire}Coupez la pierre qui vire en quatre puis détaillez-le en éclats à l'aide de la pointe d'un couteau.
\ingredient{3 \oe ufs}
\ingredient{3 c.s. d'huile d'olive}
\ingredient{10 cL de lait}Dans un saladier, fouettez les \oe ufs avec l'huile et le lait.
\ingredient{180g de farine}
\ingredient{1 sachet de levure}Mélangez la farine et la levure dans un bol puis versez dans le saladier tout en fouettant.
\ingredient{sel, poivre}
\ingredient{50g de raisains blonds}
\ingredient{100g de gruyère râpé}Salez, poivrez, ajoutez les dés de poulet, les raisins, les éclats de pierre qui vire et le gruyère râpé. Mélangez le tout délicatement.
\ingredient{une noix de beurre}Versez la préparation dans un moule à cake préalablement beurré.
\sidedish{Faites cuire au four pendant 45 minutes}.
\hint{La pierre qui vire doit être fraîche. Plus sèche, le goût est plus fort. La recette est pour 4 à 5 personnes.}
\end{recipe}
\chapitre{Plats}
\recette{Empanadas}
\ingredient{}Pour une douzaine d'empanadas
\ingredient{La pate}
\ingredient{12 cl d'eau}Dans une casserole, 
\ingredient{1/2 verre d'huile d'olive}mettre à chauffer l'eau et l'huile avec le sel.
\ingredient{une bonne pincée de sel} 
\ingredient{300g de farine} Lorsque le mélange bout, le retirer du feu et ajouter la farine en une seule fois
\ingredient{1 œuf} Lorsque la pâte forme un mélange homogène, incorporer l'œuf.
\ingredient{}Pétrir un peu la pate et la mettre de côté.
\ingredient{}Acheter de la pate brisée ça marche aussi, mais c'est moins drôle.
\ingredient{}
\ingredient{La garniture}
\ingredient{}On peut mettre tout ce qui traine : épinards, fromage, viande hachée avec des oignons, maïs, champignons...
\ingredient{}
\ingredient{}Une fois la pate étalée (il faut qu'elle soit assez fine), la découper en ronds 
\ingredient{}Déposer la garniture sur la pâte et refermer.
\ingredient{1 jaune d'œuf}Dorer chaque chausson au jaune d'œuf
\sidedish{Cuire ensuite la fournée d'empanadas au four pendant une vingtaine de minutes, à 180°}
\hint{Si les empanadas ont été faites avec soin, elles sont relativement pratiques à transporter (pour changer un peu des sandwichs...)}
\end{recipe}
\barre
\recette{Gratin de carottes aux Corn Flakes}
\ingredient{600 g. de carottes}{Éplucher les carottes et les couper en rondelles. Les faire cuire.}
\ingredient{80 g. de Corn Flakes}{Écraser les Corn Flakes en petits morceaux}
\ingredient{20 g. de beurre}{Mélanger le beurre avec les Corn Flakes}
\ingredient{1 oignon}{Dans une poële, faire revenir l'oignon avec un peu de beurre}
\ingredient{3 cuillières à soupe de farine}{Saupoudrez avec la farine, salez et poivrez la mixture}
\ingredient{30 cl. de lait}{Mouillez progressivement avec le lait et faire épaissir}
\ingredient{100 g. de fromages râpé}{Mettre dans un plat beurré le frômage et les carottes cuites, la préparation, les corn flakes, en option un peu de persil}
\cuisson{Faire cuire 15 minutes à 180 degrés.}
\end{recipe}
\barre
\recette{Gratin de carottes et pommes de terre}
\ingredient{Pour 3 personnes (et demi !)}{}
\ingredient{600 g. de pommes de terre}
\ingredient{400 g. de carottes}Eplucher les pommes de terre et les carottes puis mettre à cuire dans de l'eau jusqu'à ce que ce soit bien mou.
\ingredient{2 oignons}
\ingredient{1 gousse d'ail}
\ingredient{200 g. de lardons}Faire revenir les lardons et faire fondre les oignons et l'ail dans une poêle.
\ingredient{1 \oe uf}
\ingredient{10 cL de crème liquide}
\ingredient{Sel, poivre}Préparer une crème.
\ingredient{Emmental râpé}Dans un plat, écraser les pommes de terre et les carottes pour en faire une purée. Ajouter de l'emmental râpé et la crème, et bien tout mélanger pour avoir une préparation bien homogène.
\ingredient{}Ajouter un peu d'épices (muscade) puis saupoudrer d'emmental râpé pour que ça gratine.
\cuisson{Faire cuire à 180 degrés jusqu'à ce que le fromage du dessus soit doré.}
\end{recipe}
\barre
\recette{Gratin de courgettes Berliner Mauer}
\ingredient{}Pour 4 personnes
\ingredient{2-3 oignons}
\ingredient{1 échalote} Faire revenir les oignons et l'échalote dans une poêle avec un peu d'huile
\ingredient{400g de boeuf haché} Lorsqu'ils sont prêts, ajouter la viande hachée
\ingredient{Sel, poivre, épices} Saler, poivrer, ajouter des épices selon les goûts
\ingredient{Riz} Recouvrir le fond du plat de riz (cru)
\ingredient{3 courgettes} Laver, éplucher et couper les courgettes en rondelles
\ingredient{} Disposer une première couche de courgettes sur le riz
\ingredient{} Étaler la viande hachée lorsqu'elle est prête
\ingredient{} Rajouter une seconde couche de courgettes
\ingredient{Fromage rapé} Recouvrir de fromage rapé
\sidedish{Faire cuire une petite heure à 180° (jusqu'à ce que les courgettes soient cuites)}
\end{recipe}
\barre
\recette{Gratin de pommes de terre au jambon}
\ingredient{1.5kg de pommes de terre} En quantité raisonnable pour nourrir tout le monde
\ingredient{quelques tranches de jambon blanc} Une fois les patates épluchées, et coupées en rondelles, alterner les couches de patates, de jambon
\ingredient{de la moutarde} et de moutarde. Une préférence pour la moutarde à l'ancienne.
\ingredient{du fromage rapé}Noyer l'échafaudage de jambon et de patates sous le fromage
\ingredient{du lait ou de la crème} et rajoutez un peu de lait ou de crème pour que cela ne sèche pas
\sidedish{Au four, comme tout bon gratin qui se respecte, une heure environ à 180-200°}
\hint{Faire cuire les pommes de terre à l'eau avant de composer le plat, afin réduire le temps de cuisson au four}
\end{recipe}
\barre
\recette{Hachis parmentier revisité}
\ingredient{1.5kg de pommes de terre} En quantité raisonnable pour nourir tout le monde
\ingredient{2 oignons} Faire revenir les oignons émincés
\ingredient{500g de viande hâchée} et y rajouter la viande lorsqu'ils commencent à dorer
\ingredient{2 briquettes de coulis de tomates}Une fois que la viande est cuite, y rajouter le coulis de tomate
\ingredient{des herbes de provence, sel, poivre} assaisoner un peu puis déposer le mélange au fond d'un plat.
\ingredient{}Recouvrir avec les pommes de terre coupées en rondelles
\ingredient{du fromage rapé} Et parsemer généreusement de fromage rapé
\sidedish{Au four, comme tout bon gratin qui se respecte, une heure environ à 180-200°}
\hint{Faire cuire les pommes de terre à l'eau avant de composer le plat, afin réduire le temps de cuisson au four et manger avant 23h}
\end{recipe}

\barre
\recette{Osso bucco della nonna}
\ingredient{4 tranches de jaret de veau, environ 3cm d'épaisseur}Préparer les tranches de jaret en les ficelant comme des paquets cadeaux.
\ingredient{1 gros oignon}Couper en petits cubes.
\ingredient{1 ou 2 gousses d'ail}Couper tout fin.
\ingredient{1 kg de carottes}Couper en dés.
\ingredient{un pied de céleri branche}Enlever les feuilles et enlever les fils, et couper en rondelles.
\ingredient{Boîte de tomates pelées ou concasser (pas du jus), de 30cL environ}
\ingredient{Persil}Couper en petits morceaux (facultatif).
\ingredient{Cocotte minute en fonte (de préférence)}
\ingredient{Huile d'olive}Mettre de l'huile d'olive au fond, puis remplir la cocotte avec les légumes.
\ingredient{}Faire des petits trous dans la couche de légume et y insérer une tranche dans chacun, puis brasser un peu pour recouvrir les tranches par les légumes.
\ingredient{Sel, poivre}Saler, poivrer, et ajouter éventuellement du thym, de l'origan et du romarin.
\ingredient{4 noisettes de beurre}Ajouter une noisette sur chaque tranche pour que ca devienne un peu moelleux.
\sidedish{Fermer hermétiquement, et s'il n'y a pas de cocotte en fonte, surveiller un peu et éventuellement ajouter de l'eau (mais le moins possible). Faire cuire en deux temps : d'abord faire frémir, puis feu doux, le tout devrait prendre une heure sans toucher à rien.}
\end{recipe}
\barre
\recette{Pastel de Choclo}
\ingredient{} La version chilienne du hachis parmentier, avec du maïs. \\ Il semble que ce soit la version « étudiants chiliens fauchés », il existe une recette avec du poulet plus complète
\ingredient{5 boites de maïs} Faire une purée à partir du maïs écrasé
\ingredient{5 œufs} Y ajouter les œufs durs
\ingredient{1 poivron} Et le poivron coupé en petits moreceaux
\ingredient{500g de viande de bœuf hâchée} Faire revenir la viande
\ingredient{3 oignons} et les oignons, en assaisonnant avec sel et poivre
\ingredient{} Comme pour un hachis parmentier, déposer le mélange de viande et d'oignons au fond d'un plat beurré, et recouvrir avec le maïs
\sidedish{Au four à 200° environ une demi-heure (jusqu'à ce que le dessus soit doré)}
\end{recipe}
\barre
\recette{Polpette della nonna}
\ingredient{400g. de viande hachée}
\ingredient{15 cm de baguette de pain rassi}Enlever la croute (c'est plus facile quand c'est rassi !).
\ingredient{Eau ou lait chauds}Dans un bol, mettre l'eau ou le lait et passer au micro-ondes, puis faire tremper le pain dedans. Quand le pain est bien rempli, presser pour essorer, puis joindre à la viande hachée.
\ingredient{Sel, poivre, noix de muscade}Ajouter tout ça, puis pétrir.
\ingredient{}Laisser la préparation au frigo pendant à peu près une demie heure pour que la viande se prenne bien en masse.
\ingredient{}Préparer les boulettes de viande.
\ingredient{Farine}Déposer de la farine sur le papier alu, puis rouler les boulettes dans la farine. Les stocker quelque part :-)
\sidedish{Dans une poêle avec un peu de matière grasse chauffée, déposer les boulettes délicatement. Quand elles sont dorées, les retirer et les stocker au même endroit que tout à l'heure. Dans la même matière grasse qui est restée dans la poêle, poser un oignon (coupé en tranches fines), puis faire revenir délicatement.}
\ingredient{3 cuillères à café pleines de farine}Verser la farine dans un verre puis compléter avec de l'eau. Quand les oignons sont prêts, y verser ce verre puis vite remuer pour éviter les grumeaux. Saler un peu.
\ingredient{}Laisser mijoter un peu puis remettre les polpette, puis faire cuire à petit feu une vingtaine de minutes.
\end{recipe}
\barre
\recette{Poulet à l'ananas}
\ingredient{Une boîte de conserve d'ananas en tranche} Mettre le jus de la conserve dans une poële. Couper des morceaux d'ananas et les mettre également dans la poële. 
\ingredient{Quart de cubes de bouillon de volailles} Mettre un peu de bouillon de volailles (en poudre plutôt qu'en cube c'est bien aussi). Rajouter un peu de curry selon les goûts, et laisser mariner au moins 15 minutes (cf. documentation du bouillon de volailles).
\ingredient{Quelques cuillières de crême fraîche} Ajouter de la crème fraîche pour lier, un peu de maizena pour épaissir. 
\ingredient{Maizena ou farine}
\ingredient{4 blancs de poulet} Cela vous donne une sauce. Faites cuire à part vos blancs de poulets, accompagnez d'un peu de riz, c'est prêt. 
\hint{J'apprécie fortement cette recette car c'est facilement trouvable dans un frigo, et que l'ananas se conserve sur de très longues périodes. Idéal pour l'improviste.}
\end{recipe}
\barre
\recette{Poulet Tandoori}
\ingredient{1 yaourt nature type la laitière} Verser le yaourt dans un saladier.
\ingredient{1 pot à yaourt de vinaigre}
\ingredient{1 pot à yaourt de jus de citron}
\ingredient{1 pot à yaourt d'huile d'olive}
\ingredient{1 pot à yaourt d'épice tandoori} Mélangez tous ses ingrédients.
\ingredient{600 g. de blancs de poulet} Mettre le poulet à mariner dans les épices au min deux heures.
\sidedish{Faire griller les morceaux de poulet marinés au four ou au barbecue.}
\hint{Vous pouvez servir avec du tsatsiki et des nans.}
\end{recipe}
\chapitre{Poissons}
\recette{Gambas Grillées}
\ingredient{400 de gambas (avec tête et queue)}(Pour 2 personnes) Enlever tête, queue, carapace des gamabs.
\ingredient{1 oignon} Éplucher l'oignon et le faire revenir à la poële.
\ingredient{Huile d'olive} Quand les oignons ont une bonne tête, ajouter un filet d'huile d'olive puis les gambas et les faire cuire (jusqu'à ce qu'elles soient bien dorées).
\end{recipe}
\barre
\recette{Tartare de saumon}
\ingredient{200 g. de pavé de saumon}(Pour 2 personnes) Enlever la peau du saumon. Couper la chair du saumon en tranches très fines, puis couper les tranches en lamelles et enfin les lamelles en tout petits bouts, puis réserver.
\ingredient{2 échalotes}
\ingredient{Ciboulette} Hacher finement ces 2 trucs. Mélanger au saumon, puis saler et poivrer.
\ingredient{1 citron}
\ingredient{4 petites cuillères d'huile d'olive}
\sidedish{Mélanger le tout et laisser au frigo pendant plus d'une heure. Remélanger un quart d'heure avant de servir.}
\hint{Servir dans des assiettes avec de la salade et des tomates cerises, le tout accompagné pourquoi pas de blinis.}
\end{recipe}
\chapitre{Tartes, quiches et pizzas}
\recette{Pâte à Pizza}
\ingredient{500 g. de farine}
\ingredient{1 cuillère à café et demie de sel}
\ingredient{2 sachets de levure de boulanger}
\ingredient{3 dL d'eau tiède}
\ingredient{2 cuillères à soupe d'huile (d'olive)}
\ingredient{1 cuillère à soupe de lait chaud}
\sidedish{Cuire au four à 240°C 15 à 20 minutes}
\astuce{Utiliser de la farine fluide, ca va beaucoup plus vite pour mélanger.}
\end{recipe}
\barre
\recette{Tarte à la tomate et au comté}
\ingredient{Une pâte feuilletée (ou brisée)} Mettre la pâte dans un moule à tarte et faire des petits trous
\ingredient{De la moutarde mi-forte (3-4 grosses cuillières à soupe)} Étaler la moutarde sur le fond de la tarte
\ingredient{Du comté} Il doit y avoir assez de comté pour remplir la tarte, la limite supérieure étant une fonction affine de la taille du portefeuille. 
\ingredient{3-4 grosses tomates} Couper les tomates en rondelles et les déposer sur la pâte
\ingredient{Fromage rapé} Recouvrir le tout par du fromage rapé
\cuisson{Mettre au four à 200 degrés pendant 20 minutes environ}
\astuce{On peut remplacer le fromage par de l'emmental ou du fromage de chèvre, mais le rédacteur de la recette n'aime pas. On peut également rajouter quelques herbes selon les goûts}
\end{recipe}
\barre
\recette{Tarte aux courgettes et aux lardons}
\ingredient{10 cL de crème liquide}
\ingredient{2 \oe ufs} Mélanger les \oe ufs et la crème liquide, saler et poivrer.
\ingredient{2 ou 3 courgettes} Couper les courgettes et les faire cuire avec un filet d'huile d'olive.
\ingredient{200 g. de lardons}
\ingredient{1 oignon} Emincer l'oignon et faire rissoler les lardons et l'oignon à la poêle.
\ingredient{1 pâte brisée} Déposer les courgettes et les lardons au fond du plat, verser par-dessus la crème.
\cuisson{Jusqu'à ce que ça dore au dessus, au four à 200°C.}
\hint{On peut faire cuire les courgettes avec une tomate et des herbes de provence pour les parfumer. A mi-cuisson des courgettes, on peut aussi ajouter un soupçon de tandoori pour donner un peu de punch au plat.}
\end{recipe}
\barre
\recette{Tarte fine aux oignons et au chèvre frais}
\ingredient{3 bottes d'oignons nouveaux}Pelez les oignons en gardant un peu de tige et coupez-les en tronçons.
\ingredient{2 c.s. d'huile d'olive}Dans une poêle, faites chauffer l'huile en faites-y revenir les oignons jusqu'à ce qu'ils soient dorés, en les remuant souvent.
\ingredient{1 c.s. de sucre}
\ingredient{sel, poivre}
\ingredient{3 brins de romarin}Saupoudrez-les de sucre, salez, poivrez, ajoutez 10 à 15cL d'eau et le romarin émietté, laissez mijoter 15 minutes à découvert pour que les oignons soient tendres et caramélisés.
\ingredient{1 rouleau de pâte feuilletée}Déroulez la pâte avec son papier de cuisson sur la plaque du four (préchauffé à th. 7).
\ingredient{1 c.s. (encore) d'huile d'olive}Piquez la pâte de quelques coups de fourchette et badigeonnez-la avec l'huile d'olive. Enfournez-la pour 20 minutes environ de cuisson à blanc jusqu'à ce qu'elle soit bien dorée.
\ingredient{1 bouquet de ciboulette}Rincez, séchez et ciselez la ciboulette.
\ingredient{200 g. de fromage de chèvre frais (type Cahvroux)}Dans un bol, émiettez le chèvre à la fourchette. Mettez la tarte sur un plat de service, répartissez dessus le chèvre puis les oignons, parsemez de ciboulette ciselée et servez aussitôt.
\end{recipe}
\chapitre{Pâtes}
\recette{Fettuccine al Prosciutto}
\ingredient{500 g. de fettuccine}
\ingredient{150 g. de beurre}
\ingredient{40 g. de farine}
\ingredient{100 g. de Prosciutto}
\ingredient{crème fraiche et parmesan}
\hint{Faire fondre 100 g. + la farine}
\end{recipe}
\barre
\recette{Tagliattelle alla Bolognese}
\ingredient{500 g. de tagliatelle}
\ingredient{200 g. de viande hachée}
\ingredient{50 g. de lard}
\ingredient{1 carotte}
\ingredient{1 branche de céleri}
\ingredient{1/2 oignon}
\ingredient{25 g. de champignon}
\ingredient{1 cuillère de sauce tomate}
\ingredient{2 cuillères de crème fraiche}
\ingredient{huile, beurre, sel, poivre, parmesan, persil}
\end{recipe}
\barre
\recette{Tagliattelle aux aubergines}
\ingredient{300 g. de tagliatelle}Faire cuire et en même temps, faire la suite.
\ingredient{2 aubergines}Couper les aubergines en dés.
\ingredient{2 oignons}Emincer les oignons
\ingredient{Huile d'olive}Faire dorer les aubergines et les oignons à la poêle.
\ingredient{75 g. de pignons}
\ingredient{3 gousses d'ail}Ecraser les gousses d'ail, et les mélanger ainsi que les pignons aux aubergines quand elles ont bien cuit. Laisser un peu chauffer.
\ingredient{5 c.s. de concentré de tomate}
\ingredient{200 mL de bouillon}Verser la tomate et le bouillon dans la poêle puis laisser mijoter.
\cuisson{Mélanger la mixture de la poêle aux pâtes et laisser chauffer un peu le tout.}
\hint{On peut ajouter des olives dans le plat.}
\end{recipe}
\chapitre{Entrées}
\recette{Bruschetta}
\ingredient{1 poivron rouge} Laver et couper le poivron en deux et l'épépiner. Le placer sous le grill du four jusqu'à ce que la peau se boursoufle et noircisse. Enlever la peau et passer au mixeur.
\ingredient{3 yaourts} Dans un bol, mélanger les yaourts, saler et poivrer.
\ingredient{4 tranches de pain de campagne} Mélanger le tout et étaler sur les tranches grillées.
\end{recipe}
\barre
\recette{Camemberts frits}
\ingredient{1 camembert} (Pour 2 personnes) Couper le camembert en 2 dans l'épaisseur (ca fait 2 ronds).
\ingredient{Œufs} Battre les œufs dans un bol.
\ingredient{Chapelure} Tremper chaque demi camembert dans les œufs puis dans la chapelure, et ce deux fois pour chaque demi camembert.
\ingredient{Huile} Verser de l'huile dans une poêle et faire paner les 2 demis camemberts. Attention, il faut les poser avec la croûte vers le dessus.
\sidedish{} 3 minutes de chaque côté pour que ce soit bien doré.
\hint{A servir avec 2 cuillères de confiture de griottes dans l'assiette et des morceaux d'échalotte crue.}
\end{recipe}
\barre
\recette{Gaspacho} 
\ingredient{1 grosse boîte de tomates} En conserve, mais fraîches c'est bien aussi
\ingredient{1/2 poivron vert}
\ingredient{1/2 concombre}
\ingredient{1 échalote grise ou 1 très petit oignon coupé en 2}
\ingredient{1 gousse d'ail entière}
\ingredient{2 cuillères à soupe d'huile d'olive}
\ingredient{1/2 cuillère à café de vinaigre balsamique}
\ingredient{tabasco ou cayenne (au goût)}
\ingredient{1/2 cuillère à café de sucre}
\ingredient{sel et poivre}
\ingredient{15 cl d'eau} Mettre tous les ingrédients coupés grossièrement dans un mixer.
Démarrer l'appareil à petite vitesse et terminer à grande vitesse. Laisser le plus longtemps possible au réfrigérateur.
\\ Ajouter des glaçons au moment de servir. La recette est pour 4 personnes environ. 
\hint{Si vous êtes pressés, mettre au congélateur 30 minutes.}
\end{recipe}
\barre
\recette{Tortilla de patata}
\ingredient{1kg de pommes de terre} Éplucher et couper les patates en morceaux (pas rondelles), et les mettre à cuire dans une poêle
\ingredient{de l'huile, beaucoup d'huile} en les couvrant généreusement d'huile. Si ça ne ressemble pas à une piscine d'huile ça n'ira pas.
\ingredient{1 oignon} Rajouter l'oignon émincé, et couvrez.
\ingredient{} Une fois que les pommes de terre sont cuites, vider l'huile dans un bol et réserver le contenu de la poële dans un saladier.
\ingredient{8 œufs} Battre les œufs comme pour une omelette, saler et poivrer.
\ingredient{} Rajouter les pommes de terres/oignons aux œufs, remuer (doucement sinon ça va faire de la purée)
\ingredient{} Remettre un peu d'huile dans la poële, et faire cuire la tortilla à feu doux
\ingredient{} Une fois le premier côté doré, retourner la tortilla à l'aide d'une assiette et faire cuire l'autre côté.
\ingredient{} Et une fois que c'est cuit, manger :)
\hint{La tortilla se mange sur une tranche de pain, avec de la tomate éventuellement. Elle se mange chaude, tiède ou froide.\\ On peut également y ajouter des légumes (tomates, courgettes...) mais tout de suite ça devient un peu trop sain et équilibré.}
\end{recipe}

\barre
\recette{Tsatsiki}
\ingredient{2 concombres} Raper les concombres et les laisser bien dégorger avec du gros sel.
\ingredient{1 kg de yaourt grec}
\ingredient{1 petit bouquet d'aneth} Hacher l'aneth et la mélanger au yaourt.
\ingredient{5 c. à soupe d'huile d'olive}
\ingredient{2 c. à soupe de vinaigre} Ajouter huile et vinaigre, saler, poivrer.
\ingredient{} Ajouter les concombres et servir frais.
\hint{Vous pouvez servir avec de la pita ou en accompagnement d'agneau grillé.}
\end{recipe}
\chapitre{Crêpes et galettes}
\recette{Crêpes sucrées (avec repos)}
\ingredient{250g de farine}
\ingredient{3 \oe ufs}
\ingredient{1/2L de lait}
\ingredient{1/2 cc sel}
\ingredient{1 cs d'huile ou beurre} Mélanger tout ça et laisser reposer pendant l'après-midi.
\hint{Ajouter une cs de rhum. Tout ça fait une douzaine de crêpes.}
\end{recipe}
\barre
\recette{Crêpes sucrées (sans repos)}
\ingredient{2 mugs de farine}
\ingredient{3 mugs de lait}
\ingredient{2 \oe ufs}
\ingredient{1 sachet de sucre vanillé}
\hint{Bah celle là ne repose pas \texttt{:)}}
\end{recipe}
\barre
\recette{Galettes au sarrasin}
\ingredient{250g de farine de blé noir}
\ingredient{60g de farine blanche} Mélanger les farines.
\ingredient{1 \oe uf} Ajouter l'\oe uf.
\ingredient{3/4 L d'eau} Ajouter l'eau.
\ingredient{une pincée de sel} Ajouter la pincée de sel (vous l'auriez pas deviné).
\hint{Cela correspond à une douzaine de galettes.}
\end{recipe}
\barre
\recette{Pain perdu}
\ingredient{12 tranches de pain sec}
\ingredient{60 cL de lait}
\ingredient{3 \oe ufs}
\ingredient{Sucre en poudre}
\cuisson{Beurrer une poêle et faire griller les tranches de pain baignées dans le mélange lait/\oe ufs/sucre et servir recouvertes de sucre.}
\end{recipe}
\part*{Recettes sucrées}
\addcontentsline{toc}{part}{Recettes sucrées}
\chapitre{Biscuits}
\recette{Biscuits de noël}
\ingredient{250 g. de farine} 
\ingredient{125 g. de sucre}
\ingredient{50 g. d'amandes en poudre}
\ingredient{1 cuillière à café de levure} Mélanger la farine, le sucre, les amandes et la levure.
\ingredient{100 g. de beurre}
\ingredient{1 \oe uf} Ajouter le beurre et les \oe ufs. 
\ingredient{Des épices} Pétrir la pâte, en faire une boule. Y ajouter des épices et les zestes au choix (on peut faire plusieurs parfums, Muscade, Gingembre, Cumin, fleurs d'oranger, etc). 
\ingredient{Des zestes d'agrumes} Étaler la pâte sur du papier sulfurisé. 
\astuce{Vous pouvez étaler un peu de jaune d'\oe uf mélangé à de la canelle pour qu'ils soient dorés après cuisson. Faire les biscuits avec des formes comme un sapin, des étoiles, est également très sympa.}
\cuisson{Laisser cuire entre 7 et 10 minutes dans un four à 200 degrés.}
\end{recipe}
\barre
\recette{Cookies}
\ingredient{2 œufs}
\ingredient{200 g. de sucre roux}
\ingredient{150 g. de sucre blanc}
\ingredient{450 g. de farine}
\ingredient{1 sachet de levure}
\ingredient{200 g. de beurre ramolli (un peu moins)}
\ingredient{1 pincée de sel}
\ingredient{100 à 200 g. de pépites de chocolat}
\sidedish{Une dizaine de minute à 180 degrés par plaque (la durée dépend surtout des goûts et de leur taille)}
\hint{Remplacer le beurre et le sel par moitié beurre salé, moitié beurre} \\
Cela fait de nombreux cookies, n'hésitez pas à réduire les proportions si vous n'êtes pas si nombreux que ça (mais ils se conservent aussi très bien pour le lendemain \smiley )\\Mieux vaut mettre 180 g de beurre et beurrer + fariner une plaque, ce sera beaucoup mieux ainsi.
\end{recipe}
\barre
\recette{Rose des sables}
\ingredient{200 g. de beurre}{Faire fondre le beurre dans une casserolle à feu doux sous forme de crème}
\ingredient{300 g. de chocolat}{Rajouter le chocolat et bien remuer}
\ingredient{100 g. de sucre glace}{Ajouter le sucre glace en continuant de mélanger}
\ingredient{170 g. de Corn Flakes}{Mettre les Corn Flakes dans un grand saladier, et verser le contenu de la casserolle. Mélanger délicatement sans briser les pétales, qui doivent cependant toutes être nappées.}
\ingredient{}{Former des roses et laisser reposer au frigo jusqu'à solidification}
\hint{On peut aussi rajouter un peu de café à la préparation, du zeste d'orange\dots}
\end{recipe}
\chapitre{Gâteaux}
\recette{Clafoutis aux cerises}
\ingredient{750 g. de cerises noires}
\ingredient{150 g. de farine}
\ingredient{3 \oe ufs}
\ingredient{20 cL de lait}
\ingredient{2 c.s. de sucre}
\ingredient{1 pincée de sel}
\hint{Le lait est à mettre petit à petit.}
\sidedish{40-45 minutes à four 150 à 180 degrés. Déguster tiède.}
\end{recipe}
\barre
\recette{Crumble aux pommes}
\ingredient{4 pommes} Couper les pommes en petits morceaux et les étaler dans un plat.
\ingredient{100 g. de beurre}
\ingredient{200 g. de sucre}
\ingredient{225 g. de farine} Mélanger tout ca, en ayant fait un peu fondre le beurre à l'air libre (30 minutes), jusqu'à obtenir un mélange poudreux.
\ingredient{} Verser une fine pellicule de sucre sur les pommes et enfourner à 180°C jusqu'à ce que les pommes soient bien moelleuses (le jus commencera à bouillir au fond du plat).
\ingredient{} Verser la poudre sur les pommes puis réenfourner le tout jusqu'à ce que le biscuit soit doré.
\hint{Ca fera un crumble pour 3 personnes qui ont faim, donc ne pas hésiter à faire plus gros. Avec des fraises dessus, c'est encore meilleur !}
\end{recipe}
\barre
\recette{Fondant chocolat-crême de marron}
\ingredient{250 g. de crême de marron}
\ingredient{50 g. de chocolat dessert} Faire fondre le beurre et le choclat cassé en carreaux au micro-onde. Il faut bien surveiller et mélanger fréquemment. 
\ingredient{50 g. de beurre demi sel} Ajouter la crême de marron, bien battre. Ajouter les œufs en continuant de battre énergiquement. 
\ingredient{2 œufs} 
\sidedish{Faire cuire environ 30 minutes dans un grand moule. Le gâteau doit être fondant mais pas coulant. } \\ 
Attention il s'agit d'une recette pour trois ou quatre personnes seulement. 
\end{recipe}
\barre
\recette{Fudgy brownies}
\ingredient{200 g. de chocolat}
\ingredient{130 g. de beurre} Faire fondre le chocolat et le beurre ensemble.
\ingredient{6 œufs}
\ingredient{300 g. de sucre}
\ingredient{100 g. de sucre de canne}
\ingredient{1 sachet de sucre vanillé}
\ingredient{1 pincée de sel}
\ingredient{180 g. de farine} Fouetter le tout jusqu'à obtention d'un mélange léger.
\ingredient{} Rajouter le mélange beurre-chocolat.
\ingredient{100 g. de noix} Hacher les noix et les ajouter.
\sidedish{Faire cuire pendant 45min à 200°C.}
\hint{On peut remplacer les noix par des noisettes ou des amandes}
\end{recipe}
\barre
\recette{Gâteau à l'orange}
\ingredient{150 g. de beurre ramoli} 
\ingredient{150 g. de sucre}
\ingredient{3 œufs} Mélanger le beurre, le sucre et les œufs. 
\ingredient{150 g. de farine}
\ingredient{1 sachet de levure chimique} Y rajouter la farine et la levure. 
\ingredient{3 oranges} Prendre le zeste de deux oranges et l'ajouter à la pâte. Transformer en jus deux des oranges et ajouter ce jus au gâteau. 
\ingredient{} Beurrer un plat et y verser la pâte. 
\cuisson{Faire cuire une vingtaine de minutes à 180 degrés, de préférence en chaleur tournante} 
\ingredient{} Prendre le jus de l'orange restante et y ajouter un peu de sucre. Faites le chauffer légèrement pour obtenir un sirop. 
\ingredient{} Verser ce sirop sur le gâteau à la sortie du four. Vous pouvez ensuite y ajouter des petits morceaux de chocolats, à la fois pour la décoration et pour le goût. 
\astuce{Un peu de citron, que ce soit pour les zestes ou pour le sirop, ça ne fait pas de mal. Attention cependant à ne pas trop en mettre sous peine de rendre le gâteau trop acide}
\end{recipe}
\barre
\recette{Gâteau au chocolat (maman de Madeleine)}
\ingredient{200 g. de chocolat}
\ingredient{200 g. de beurre} Faire fondre le chocolat et le beurre au bain-marie.
\ingredient{5 œufs} Battre les œufs
\ingredient{200 g. de sucre}
\ingredient{80 g. de farine} Incorporer le sucre, la farine et la préparation.
\end{recipe}
\barre
\recette{Gateaux aux poires}
\ingredient{800g de poires (à la louche)} Peler les poires, enlever les trognons (c'est meilleur) les couper en morceaux pas trop petits.
\ingredient{4 œufs}
\ingredient{250 g. de sucre} Mélanger les œufs avec 200g de sucre, en remuant jusqu'à ce que ça blanchisse
\ingredient{125 g. de beurre} Rajouter le beurre fondu
\ingredient{250 g. de farine} la farine
\ingredient{1 sachet de levure} sans oublier la levure
\ingredient{de la cannelle} un peu de cannelle
\ingredient{de la vanille ou de l'extrait de vanille} et de vanille parce que c'est quand même bon
\ingredient{} Rajouter les poires en morceaux
\ingredient{} une fois la pate dans le plat, saupoudrer le dessus du reste du sucre 
\sidedish{Au four, environ une demi-heure, th6/150° (à surveiller quand même)}
\end{recipe}

\barre
\recette{Gâteau aux pommes}
\ingredient{125g. de sucre}
\ingredient{3 œufs}
\ingredient{1 sachet de sucre vanillé}
\ingredient{125g. de farine}
\ingredient{1 tasse de lait}
\ingredient{125g. de beurre fondu}
\ingredient{1 tasse à café de rhum}
\ingredient{1 paquet de levure}
\ingredient{4 ou 5 pommes}
\hint{Utiliser des goldy ou des reinettes}
\sidedish{Cuisson au four, th. 7 puis 6 (35 à 40 minutes), dans un plat de 24 ou 26 cm de diamètre}
\end{recipe}
\barre
\recette{Gâteau roulé}
\ingredient{4 œufs}
\ingredient{90 g. de sucre} Mélanger les jaunes et le sucre
\ingredient{1 sachet de sucre vanillé}
\ingredient{80 g. de farine} Ajouter la farine puis les blancs battus en neige
\ingredient{} Faire cuire sur une plaque légèrement farinée et beurrée à Th. 5 pendant 10 minutes.
\ingredient{} Dérouler sur un linge humide (ou une feuille de papier alu/cuisson) puis confiturer et rouler.
\end{recipe}
\barre
\recette{Gâteaux aux marrons}
\ingredient{1 œuf}
\ingredient{65 g. de sucre}
\ingredient{1 cuillère à soupe de rhum brun}
\ingredient{1 pincée de sel} Mélanger ces premiers ingredients. On appelle ça le mélange A.
\ingredient{65 g. de farine}
\ingredient{1 cuillère à café de levure} Ajouter ça au mélange A. On va s'attaquer maintenant au mélange B.
\ingredient{65 g. de beurre} Faire ramollir le beurre (un peu, et avec la spatule)
\ingredient{135 g. de crème de marrons} Ajouter les marrons au mélange B.
\ingredient{} Mélanger A et B.
\sidedish{Faire cuire dans un moule beurré et fariné 40 minutes à 180°C}
\end{recipe}
\barre
\recette{Gâteau au yaourt (Nonni)}
\ingredient{1 yaourt}
\ingredient{4 pots de farine}
\ingredient{3 pots de sucre}
\ingredient{1 pot d'huile}
\ingredient{3 œufs}
\ingredient{1 cuillère à café de levure}
\ingredient{1 paquet de sucre vanillé}
\hint{Ajouter au choix du limoncello, de la mandarine Napoléon ou des zests d'orange.}
\sidedish{Cuisson à th. 5 pendant un quart d'heure puis 20 minutes à four bas.}
\end{recipe}
\barre
\recette{Gâteau au yaourt (Bertrand)}
\ingredient{1/2 paquet de levure}
\ingredient{1 pot de yaourt}
\ingredient{1/2 pot d'huile}
\ingredient{2 pots de sucre}
\ingredient{3 pots de farine}
\ingredient{2 œufs}
\ingredient{1 zeste de citron}
\ingredient{} Mélanger dans l'ordre
\hint{On peut ajouter 100 g. de pépites de chocolat, sinon c'est trop léger.}
\sidedish{30 minutes à th. 6.}
\end{recipe}
\barre
\recette{Moelleux au chocolat (presque marmiton)}
\ingredient{200 g. de chocolat} Faire fondre le chocolat à feu doux dans une casserolle avec un peu d'eau
\ingredient{125 g. de beurre} Un coup de micro-onde pour le rendre liquide
\ingredient{4 œufs} 
\ingredient{225 g. de sucre} Mélanger le sucre, le beurre et les œufs 
\ingredient{125 g. de farine} 
\ingredient{1/2 sachet de levure chimique} Rajouter lentement la farine et la levure. Terminer par le chocolat. Beurrer le moule et verser la préparation
\sidedish{30 minutes à 180° (th. 6)}
\hint{Cela fonctionne aussi avec un moule à cake, il suffit de laisser un peu moins cuire et le c\oe ur reste un peu fondant. Soupoudrez un peu de sucre glace peut encore améliorer le gâteau.}
\end{recipe}
\barre
\recette{Muffins aux raisins}
\ingredient{2 poignées de raisins secs} Mettre les raisins secs à tremper dans un bol d'eau chaude
\ingredient{120g de sucre}
\ingredient{100g de beurre} Mélanger le suvre et le beurre fondu dans un saladier
\ingredient{2 \oe ufs} Rajouter 2 \oe ufs
\ingredient{120g de farine} Rajouter la farine
\ingredient{1 caf de levure} et une cuillère à café de levure
\ingredient{Cannelle, gingembre ou mélange 4 épices} Rajouter les épices et les raisins secs
\sidedish{Environ 20min de cuisson dans un four préchauffé à 180°.}
\end{recipe}
\barre
\recette{Nems sucrés (recette de redisdead)}
\ingredient{1 pomme, 2 poires ou 2 bananes}
\ingredient{Pépites de chocolat}Préparer les fruits, les peler et les découper en petits morceaux.
\ingredient{6 feuilles de brick}Découper chaque feuille de brick en deux, puis déposer des fruits le long de la partie coupée, au centre, sur 5 centimètres. Replier les pointes des feuilles vers le centre, puis la rouler pour faire des nems.
\ingredient{Un jaune d'\oe uf}
\ingredient{Un sachet de sucre vanillé}Une fois les nems prêts, les déposer sur une plaque en teflon ou sur du papier cuisson et les laquer avec un pinceau alimentaire à l'aide d'un mélange de sucre vanillé et de jaune d'\oe uf.
\sidedish{7 à 10 minutes à 180° (th. 6), juste le temps de les faire dorer.}
\hint{On peut les servir avec une salade de fruits, de la glace ou une compote. Compter 4 nems maxi par personne pour un dessert.}
\end{recipe}
\barre
\recette{Saucisson surprise au chocolat}
\ingredient{Une boîte de 30 biscuits à la cuillère}
\ingredient{4 c.s. de cacao non sucré} 
\ingredient{Une quinzaine de noix} Découper les noix en morceaux
\ingredient{100 g. de beurre mou}  
\ingredient{4 c.s. de lait} Écraser les biscuits dans un saladier, ajouter le cacao, les noix, le beurre et le lait. Bien mélanger. Ramasser le tout pour donner la forme d'un saucisson.
\ingredient{3 c.s. de sucre glace}
\sidedish{3h au frigo, dans du papier alu. A la sortie, rouler le saucisson dans le sucre glace et servir en tranches.}
\end{recipe}
\barre
\recette{Stollen}
\ingredient{100g de farine} Délayer la farine dans le lait tiède et y ajouter la levure. 
\ingredient{100mL de lait}  Laisser reposer 30minutes.
\ingredient{Un sachet de levure de boulangerie} Battre le beurre mou avec le sucre.
\ingredient{165g de beurre} Y ajouter l'oeuf, le jaune puis la cardamome et les clous de girofle broyés au mortier, ainsi que la tonka rapée, la vanille, la cannelle, la muscade.
\ingredient{70g de sucre} Former un puit et y ajouter la farine, le rhum et commencer à malaxer, ajouter si du lait pour obtenir une pâte souple et bien pétrir.
\ingredient{1 \oe uf et un jaune} Ajouter les raisins, les cramberries et les amandes et pétrir de nouveau.
\ingredient{3 capsules de cardamome} Mettre dans un saladier, couvrir d'un linge propre, humide et laisser lever pendant une heure.
\ingredient{2 clous de girofle} Pétrir de nouveau un peu et remettre à lever 30minutes.
\ingredient{1/2 fève tonka rapée} Façonner un boudin de 30cm avec la pâte d'amande, au besoin l'assouplir avec un peu de rhum.
\ingredient{1CC d'extrait de vanille} Étaler la pâte à stollen sur une longueur de 30cm, mettre la pâte d'amande dans la longueur et refermer. 
\ingredient{1CC de cannelle en poudre} Placer la soudure en dessous et enfourner 30minutes à 190°C.
\ingredient{1/2CC de muscade rapée} Saupoudrer de sucre glace en sortant du four et laisser refroidir.
\ingredient{400g de farine}
\ingredient{3CS de rhum brun}
\ingredient{Lait}
\ingredient{85g de raisins secs}
\ingredient{50g de cramberries séchées}
\ingredient{50g d'amandes}
\ingredient{250g de pâte d'amande}
\end{recipe}
\barre
\recette{Tiramisu à l'orange}
\ingredient{2 oranges}
\ingredient{Biscuits à la cuillère}Faire tremper les biscuits dans le jus d'orange.
\ingredient{25 mL de miel liquide}
\ingredient{6 sachets de sucre vanillé}
\ingredient{25 cL de crème liquide bien fraîche}Monter la crème en Chantilly, ajouter le miel et le sucre vanillé, mélanger.
\ingredient{250 g. de mascarpone}Rajouter le mascarpone et mélanger.
\hint{Vous pouvez au moment de servir ajouter des fraises.}
\end{recipe}
\chapitre{Mousses et crèmes}
\recette{Crème anglaise}
\ingredient{1 L de lait}
\ingredient{2 paquets de sucre vanillé} Faire bouillir le lait avec le sucre vanillé.
\ingredient{100 g. de sucre}
\ingredient{8 jaunes d'\oe ufs} Mélanger sucre et jaunes d'\oe ufs.
\ingredient{} Mélanger le tout. Faire chauffer en remuant, sans laisser bouillir.
\hint{Accompagner d'\oe ufs à la neige (8 blancs d'\oe uf + 150 g. de sucre}
\end{recipe}
\barre
\recette{Crème aux pêches}
\ingredient{1/2 litre de crème patissière} Préparer la crème patissière et la mettre au frigo
\ingredient{Boite de pêches en sirop} Egoutter la boite en gardant le sirop
\ingredient{Citron} Ajouter le jus d'un citron et un peu de sirop, puis passer le tout au mixer
\ingredient{} Mélanger à la crème patissière refroidie, mettre dans des coupes puis faire refroidir au frigo.
\end{recipe}
\barre
\recette{Mousse au chocolat}
\ingredient{4 œufs} Séparer les jaunes des blancs (à battre en neige)
\ingredient{125 g. de chocolat}
\ingredient{125 g. de sucre}
\ingredient{125 g. de beurre}
\end{recipe}
\barre
\recette{Mousse pralinée}
\ingredient{200 g. de pralinoise (Poulain)} Faire fondre la pralinoise. Elle doit rester tiède (environ 35 degrés).
\ingredient{2,5 dL de crème fraiche liquide} Battre la crème en chantilly et la laisser au frais.
\ingredient{3 blancs d'\oe ufs} Battre les blancs en neige.
\ingredient{60 grammes de sucre semoule} Quand ils sont bien fermes, ajouter le sucre ; Incorporer ensuite délicatement le chocolat fondu  puis de la même manière la crème chantilly.
\ingredient{} Mettre dans un saladier et garder au frais 2h minimum.
\hint{Juste avant de servir, saupoudrer de cacao non sucré et de pralines broyées}
\end{recipe}
\chapitre{Cakes}
\recette{Cake canelle et amandes}
\ingredient{225 g. de beurre}
\ingredient{75 g. de sucre blanc}
\ingredient{50g de sucre roux} Mélanger le beurre et les sucres
\ingredient{100 g. de poudres d'amandes} 
\ingredient{225 g. de farine}
\ingredient{1/2 sachet de levure}
\ingredient{4 épices} Quantité selon les goûts, environ 2 cuillière à café
\ingredient{1 cuillière à café de canelle} Y ajouter un à un tout les ingrédients 
\ingredient{7,5 cl de crême fraîche} Ajouter la crème pour fluidifier la pâte
\sidedish{Beurrer le moule et laisser cuire 40 minutes à 180 degrés (th. 6).}
\end{recipe}
\chapitre{Macarons et ganaches}
\recette{Ganache au chocolat (à réaliser la veille)}
\ingredient{250 g. de chocolat noir} Casser le chocolat en petits morceaux dans un saladier.
\ingredient{20 cL de crème liquide} Porter la crème à ébullition, la verser sur le chocolat jusqu'à ce qu'il ait complétement fondu.
\ingredient{70 g. de beurre} Ajouter le beurre coupé en petits morceaux et faire fondre en remuant.
\ingredient{} Laisser refroidir à température ambiante puis réserver au réfrigérateur.
\end{recipe}
\barre
\recette{Macarons (environ $50$ macarons de $3cm$ de diamètre)}
\ingredient{200 g. de sucre glace}
\ingredient{200 g. de poudre d'amandes} Mixez longuement afin d'obtenir une poudre fine et homogène. Passer au tamis et mettre de côté. Cette poudre s'appelle le "tant pour tant"
\ingredient{200 g. de sucre en poudre}
\ingredient{8 cL d'eau} Dans une casserole, portez l'eau et le sucre en poudre à ébullition. La température ne doit pas dépasser 110°C.
\ingredient{80 g. de blanc d'\oe uf (environ 3)} Montez les blancs en neige pas trop ferme.
\ingredient{} Versez le sucre cuit sur les blancs en un fin filet en laissant tourner le batteur (à faible vitesse) jusqu'à ce que la meringue ait quasiment refroidi. C'est long.
\ingredient{80 g. de blanc d'\oe uf} Mélangez au "tant pour tant" afin d'obtenir une épaisse pâte d'amande.
\ingredient{} Ajoutez le colorant à la pâte d'amande.
\ingredient{} Le macaronage: incorporer une petite quantité de meringue à la pâte d'amande à l'aide d'une spatule souple en un geste régulier du fond vers le haut et des bords vers le centre. Incorporez en une fois le reste de la meringue aussi délicatement.
\ingredient{} Réalisez des petites boules de pâte bien espacées et bien calibrées sur du papier sulfurisé.
\ingredient{} Tapotez doucement la plaque par le dessous afin de faire éventuellement ressortir des petites bulles d'air et d'uniformiser les coques.
\ingredient{} Mettre les plaques de côté pendant une heure environ afin de permettre aux coques de sécher légèrement.
\sidedish{Enfournez 13min à 145°C (15min pour des macarons de 8cm de diamètre)}
\sidedish{} À la sortie du four glisser la feuille de papier sulfurisé sur un plan de travail légèrelent humidifié pour arrêter la cuisson plus rapidement.
\hint{Vous pouvez également utiliser des moules souples pour faire cuire les macarons et ne pas les faire reposer une heure avant de les enfourner.}
\hint{Pour des colorants variés : www.argato.com}
\end{recipe}
\chapitre{Tartes}
\recette{Tarte Tatin}
\ingredient{8 pommes golden} (ou tout autre pommes du même genre).
\ingredient{1 rouleau de pâte feuilletée} Éplucher les pommes et les couper en moitié ou quartier. Enlever leurs coeurs. 
\ingredient{70g de beurre} Faire fondre le beurre (dans une poële par exemple).
\ingredient{50g de sucre roux} Rajouter les sucres au beurre, et faire cuire doucement en caramel .
\ingredient{50g de sucre blanc} Mettre le caramel au fond d'un plat à tarte. Rajouter par dessus les morceaux de pommes. 
\ingredient{Un peu de cannelle} Soupoudrez les pommes de cannelle.
\ingredient{2 sachets de sucre vanillée} Soupoudrez avec le sucre vanillé. Recouvrir de pâte feuilletée en prenant garde à bien appuyer sur les bords.
\cuisson{Laisser au four du 30-40 minutes à 180 degrés.}
\hint{Au cours de la cuisson, vous pouvez appuyez avec une spatule pour "écraser" un peu les pommes.}
\end{recipe}
\part*{Mixtures}
\addcontentsline{toc}{part}{Mixtures}
\chapitre{Shooters}
\recette{B52}
\ingredient{1 dose de kehlua}
\ingredient{1 dose de bailey}A ajouter sur le kehlua à la cuillère.
\ingredient{1 peu de grand marnier}A ajouter sur le bailey à la cuillère.
\ingredient{}Flamber le grand marnier.
\end{recipe}
\barre
\recette{Hypermétrope}
\ingredient{1 dose de vodka}
\ingredient{1 dose de chartreuse}
\end{recipe}
\barre
\recette{Orgasme}
\ingredient{1 dose de Get27}
\ingredient{1 dose de Bailey}A ajouter sur le Get27 à la cuillère.
\end{recipe}
\chapitre{Jus de fruits}
\recette{Coup de jus (Helixir)}
\ingredient{Banane}
\ingredient{Fraise}
\ingredient{Orange}
\end{recipe}
\barre
\recette{Fou de jus (Helixir)}
\ingredient{Ananas}
\ingredient{Framboise}
\ingredient{Feuilles de menthe}
\ingredient{Jus de pomme}
\end{recipe}
\barre
\recette{Tropical Storm (Helixir)}
\ingredient{Fruit de la passion}
\ingredient{Mangue}
\ingredient{Orange}
\end{recipe}
\chapitre{Cocktails}
\recette{Black Russian}
\ingredient{3 doses de vodka}
\ingredient{1 dose 1/2 de kahlua}Verser directement sur des glaçons.
\end{recipe}
\barre
\recette{Bloody Mary}
\ingredient{2 doses de vodka}
\ingredient{Le triple en jus de tomate}
\ingredient{Un peu de jus de citron}
\ingredient{2 gouttes de tabasco}
\ingredient{Sel, poivre}
\hint{On peut ajouter de la sauce worcestershire et du sel de céleri.}
\end{recipe}
\barre
\recette{Blue Lagoon}
\ingredient{2 doses de vodka}
\ingredient{1 dose de jus de citron}
\ingredient{Un peu de curaço bleu}
\hint{Mélanger au shaker et servir sur des glaçons}
\end{recipe}
\barre
\recette{Cocktail glacial}
\ingredient{2 doses de vodka}
\ingredient{Schweppes tonic}
\ingredient{Un peu de curaço bleu}
\hint{Servir avec des glaçons}
\end{recipe}
\barre
\recette{Cuba libre}
\ingredient{3 doses de rhum}
\ingredient{2 doses de jus de citron vert}
\ingredient{Coca-cola}
\hint{Verser sur des glaçons.}
\end{recipe}
\barre
\recette{Gin Fizz}
\ingredient{3 doses de gin}
\ingredient{2 doses de jus de citron}
\ingredient{1 dose de sirop de sucre de canne}
\ingredient{Eau gazeuse}
\hint{Mélanger tout sauf l'eau dans un shaker et servir sur des glaçons ; décorer d'une tranche de citron vert.}
\end{recipe}
\barre
\recette{Gin tonic}
\ingredient{2 doses de gin}
\ingredient{Tonic}
\hint{Servir sur des glaçons}
\end{recipe}
\barre
\recette{Mixed kir}
\ingredient{Cardinal}1/5 crème de cassis, 4/5 vin rouge frais, beaujolais par exemple.
\ingredient{Kir}1/6 crème de cassis, 5/6 vin blanc frais, mâcon par exemple.
\ingredient{Blanc cassé}1/8 crème de cassis, 7/8 vin blanc.
\ingredient{Dry kir}1/6 gin, 5/6 vin blanc, 1 trait de crème de cassis.
\hint{Servir dans des verres à vin de plus en plus petit et dans l'ordre des couleurs pour faire un dégradé.}
\end{recipe}
\barre
\recette{Mojito Chartreuse}
\ingredient{1/4 de citron vert}Ecraser dans un verre le citron.
\ingredient{2 c.c. de sucre}Ajouter le sucre et écraser encore.
\ingredient{10 feuilles de menthe frâiche}Ajouter la menthe en gardant la tige.
\ingredient{2 doses de chartreuse verte}Verser la chartreuse.
\ingredient{Galçons}Garnir de glaçons.
\ingredient{Eau gazeuse}Compléter le verre et le boire.
\end{recipe}
\barre
\recette{Mojito}
\ingredient{1, 2 ou 3 doses de rhum}
\ingredient{Feuilles de menthe}Ajouter les feuilles de menthe, non coupées, et les piler.
\ingredient{1/2 citron vert}Ajouter le jus du citron.
\ingredient{Glaçons ou glace pilée}Ajouter la glace.
\ingredient{Eau gazeuse}Ajouter l'eau gazeuse jusqu'à remplir le verre.
\hint{Plein de variantes possibles... on peut remplacer l'eau gazeuse par de la limonade (auquel cas pas besoin du jus de citron, seulement une lamelle à poser à la surface) ou par du tonic. On peut aussi sucrer (pas plus de 2 cuillères à café par verre). On peut également ajouter du sucre sur les bords du vert avant de présenter le cocktail.}
\end{recipe}
\barre
\recette{Pina Colada}
\ingredient{2 doses de rhum blanc}
\ingredient{1 dose de rhum brun}
\ingredient{2 doses de lait de coco}
\ingredient{Jus d'ananas}
\hint{Mixer au mixeur avec des glaçons.}
\end{recipe}
\barre
\recette{Planteur}
\ingredient{3L d'ananas}
\ingredient{6L d'orange}
\ingredient{3L de rhum blanc}
\ingredient{1L de rhum vieux}
\ingredient{2L de sirop de canne}
\ingredient{10 clous de girofle}
\ingredient{3 cuillères à café de cannelle}
\ingredient{Noix de muscade râpée}
\ingredient{1kg de citrons}
\ingredient{Angustura}
\end{recipe}
\barre
\recette{Punch}
\ingredient{33 cL de jus d'orange}
\ingredient{33 cL de jus d'ananas}
\ingredient{33 cL de nectar de goyave}
\ingredient{3 dL de rhum blanc}
\ingredient{3 dL de rhum brun}
\ingredient{2 dL de sucre de canne liquide}
\ingredient{Muscade}
\ingredient{Canelle}
\ingredient{Vanille en gousse}
\ingredient{} Préparer 24h à l'avance et mettre au réfrigérateur.
\hint{Vous pouvez remplacer les différents jus de fruit par 1L de jus multivitaminé.}
\end{recipe}
\barre
\recette{Sex on the beach}
\ingredient{1 dose de vodka}
\ingredient{1 dose de liqueur de melon}
\ingredient{1 dose de chambord}
\ingredient{3 doses de jus d'ananas}
\ingredient{3 doses de jus de canneberge}
\end{recipe}
\barre
\recette{Tequila sunrise}
\ingredient{3 doses de tequila}
\ingredient{1 dose de sirop de grenadine}
\ingredient{Jus d'orange}
\hint{Mettre le sirop en dernier pour obtenir un dégradé, et servir avec une rondelle d'orange.}
\end{recipe}
\barre
\recette{Vesper}
\ingredient{3 doses de gin Gordon}
\ingredient{1 dose de vodka}
\ingredient{1/2 dose de Kina Lillet}
\ingredient{1 grand zest de citron}
\hint{On peut remplacer le Kina par du Vermouth/Martini, et servir avec une olive pour décorer.}
\end{recipe}
\barre
\recette{Vodka Martini}
\ingredient{3 doses de vodka}
\ingredient{1 dose de Martini}Mélanger sur des glaçons pour refroidir, puis verser dans un verre sans glaçons.
\hint{Servir avec une ou deux olives dénoyautées.}
\end{recipe}
\chapitre{Infusions}
\recette{Grog hivernal}
\ingredient{Rhum (brun de préférence)}Verser le rhum dans un mug.
\ingredient{Jus d'un demi-citron}Verser le citron dans le rhum.
\ingredient{Rhum}Boire le rhum qui reste dans la bouteille (facultatif).
\ingredient{Eau}Faire chauffer l'eau (chaude mais pas bouillante).
\ingredient{Eau chaude}Verser l'eau dans le mug.
\ingredient{Miel}Ajouter un peu de miel pour adoucir le tout.
\end{recipe}
\barre
\recette{Thé glacé à la mirabelle (recette de redisdead)}
\ingredient{Thé à la mirabelle}
\ingredient{80cL d'eau chaude}Faire infuser du thé à la mirabelle pour 3 personnes dans l'eau chaude (pas bouillante), pendant 5 minutes.
\ingredient{Une douzaine de glaçons}Ajouter les glaçons et laisser refroidir le thé pendant une heure.
\ingredient{Sirop de mirabelle}Mélanger ensuite 7 cuillères à soupe de sirop de mirabelle dans le thé refroidi. Placer le récipient initial au congélateur pendant 1 heure en vérifiant que le thé ne gèle pas.
\end{recipe}
